\documentclass[11pt]{article}

\pdfobjcompresslevel=3    % compress PDF objects
\pdfcompresslevel=9       % compress streams more (0..9)
\pdfminorversion=4


\usepackage[utf8]{inputenc} % remove if using XeLaTeX or LuaLaTeX
\usepackage[T1]{fontenc}

% Layout & graphics
\usepackage[a4paper, total={7.5in, 9in}]{geometry}
\usepackage{graphicx}              % images
\usepackage[dvipsnames,table]{xcolor} % single xcolor invocation (keeps table option)

% Math & symbols
\usepackage{amsmath}
\usepackage{amssymb}
\usepackage{gensymb}

% Typography & layout helpers
\usepackage{microtype}
\usepackage{float}                 % [H] placement
\usepackage{caption}
\usepackage{subcaption}            % modern subfigure support (do NOT load subfig)
\usepackage{multirow}
\usepackage{titlesec}

\usepackage{lmodern}
\usepackage{microtype}

\definecolor{xppblue}{RGB}{37, 150, 190}

\begin{document}

\begin{table}[h]
\centering
\scalebox{1.5}{
\begin{tabular}{|ccll|cl|}
\hline
\multicolumn{4}{|c|}{ZADANIE Z WYKŁADU}                                                                                                                            & \multicolumn{2}{c|}{\multirow{2}{*}{\begin{tabular}[c]{@{}c@{}}rok akademicki:\\ 2025/26\end{tabular}}} \\ \cline{1-4}
\multicolumn{4}{|c|}{\textbf{Układy elektroniki użytkowej}}                                                                                                                          & \multicolumn{2}{c|}{}                                                                                   \\ \hline
\multicolumn{4}{|c|}{\multirow{2}{*}{\textit{Ograniczniki napięcia}}}                                                                                           & \multicolumn{2}{c|}{\multirow{2}{*}{poniedziałek 9:45}}                                                \\
\multicolumn{4}{|c|}{}                                                                                                                                                       & \multicolumn{2}{c|}{}                                                                                   \\ \hline
\multicolumn{1}{|c|}{WARiE, AiR, sem 5}          & \multicolumn{3}{c|}{\multirow{2}{*}{\begin{tabular}[c]{@{}c@{}}  Jan Andrzejewski\\\end{tabular}}} & \multicolumn{2}{l|}{\multirow{2}{*}{Punkty:}}                                                           \\ \cline{1-1}
\multicolumn{1}{|c|}{2.11.2025}                   & \multicolumn{3}{c|}{}                                                                                                   & \multicolumn{2}{l|}{}                                                                                   \\ \hline
\end{tabular}
}
\end{table}
\vspace{2\baselineskip}

\section{Układ z diodami Zenera w gałęzi sprzężenia zwrotnego}

\subsection{Rezystancje w torze sprzężenia zwrotnego}

\begin{figure}[H]
    \centering
    \includegraphics[width=0.5\linewidth]{img/schamtostry.png}
    \caption{Ogranicznik "ostry"}
\end{figure}

\begin{figure}[H]
    \centering
    \includegraphics[width=0.5\linewidth]{img/ostryprzebieg.png}
    \caption{przebieg z ostrym odcięciem}
\end{figure}

Na przebiegu można zauważyc najistotniejszy moment pracy ogranicznika. Wypłaszczenie w sygnale wyjsciowym ma miejsce po osiągnieciu przez niego wartosci około 5,8 V na te wartosc składa sie napiecie Zenera 5,1V i 0,7V czyli spadek na diodzie krzemowej kiedy prad płynie w kierunku przewodzenia.

\begin{figure}[H]
    \centering
    \includegraphics[width=0.5\linewidth]{img/scheamtklasyk.png}
    \caption{przebieg z ostrym odcięciem}
\end{figure}

W tym wariancie do pętli sprzężenia zwrotnego dodano rezystor o wartości 10.
Element ten został włączony szeregowo z gałęzią zawierającą diody Zenera, co znacząco wpływa na charakterystykę pracy układu. Obecność rezystora powoduje, że ograniczanie napięcia
nie następuje tak gwałtownie jak w wariancie „ostrym”. Przejście z obszaru liniowego do nasycenia
jest bardziej płynne, a amplituda sygnału wyjściowego zmienia się stopniowo w pobliżu napięcia
progowego.


\subsection{Czy próg zadziałania ogranicznika zależy od wzmocnienia układu?}
Tak wzmocnienie ma znaczenie ze względu na to że, mniejszym sygnałem wejściowym uda sie przekroczyć próg napięcia Zenera i osiągnąć przewodzenie w torze diod.

\subsection{Ograniczenie ze wzgledu na źródło zasilania}




\end{document}