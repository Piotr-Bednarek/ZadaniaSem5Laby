\documentclass[11pt]{article}

\pdfobjcompresslevel=3    % compress PDF objects
\pdfcompresslevel=9       % compress streams more (0..9)
\pdfminorversion=4


\usepackage[utf8]{inputenc} % remove if using XeLaTeX or LuaLaTeX
\usepackage[T1]{fontenc}

% Layout & graphics
\usepackage[a4paper, total={7.5in, 9in}]{geometry}
\usepackage{graphicx}              % images
\usepackage[dvipsnames,table]{xcolor} % single xcolor invocation (keeps table option)

% Math & symbols
\usepackage{amsmath}
\usepackage{amssymb}
\usepackage{gensymb}

% Typography & layout helpers
\usepackage{microtype}
\usepackage{float}                 % [H] placement
\usepackage{caption}
\usepackage{subcaption}            % modern subfigure support (do NOT load subfig)
\usepackage{multirow}
\usepackage{titlesec}

\usepackage{lmodern}
\usepackage{microtype}

\definecolor{xppblue}{RGB}{37, 150, 190}

\begin{document}

\begin{table}[h]
\centering
\scalebox{1.5}{
\begin{tabular}{|ccll|cl|}
\hline
\multicolumn{4}{|c|}{SPRAWOZDANIE Z LABORATORIUM}                                                                                                                            & \multicolumn{2}{c|}{\multirow{2}{*}{\begin{tabular}[c]{@{}c@{}}rok akademicki:\\ 2025/26\end{tabular}}} \\ \cline{1-4}
\multicolumn{4}{|c|}{\textbf{Układy elektroniki użytkowej}}                                                                                                                          & \multicolumn{2}{c|}{}                                                                                   \\ \hline
\multicolumn{4}{|c|}{\multirow{2}{*}{\textit{Generator RC z mostkiem Wiena}}}                                                                                           & \multicolumn{2}{c|}{\multirow{2}{*}{czwartek 8:00}}                                                \\
\multicolumn{4}{|c|}{}                                                                                                                                                       & \multicolumn{2}{c|}{}                                                                                   \\ \hline
\multicolumn{1}{|c|}{WARiE, AiR, sem 5}          & \multicolumn{3}{c|}{\multirow{2}{*}{\begin{tabular}[c]{@{}c@{}} \underline{1. Jan Andrzejewski}\\2. Mateusz Banaszak\\\end{tabular}}} & \multicolumn{2}{l|}{\multirow{2}{*}{Punkty:}}                                                           \\ \cline{1-1}
\multicolumn{1}{|c|}{27.11.2025}                   & \multicolumn{3}{c|}{}                                                                                                   & \multicolumn{2}{l|}{}                                                                                   \\ \hline
\end{tabular}
}
\end{table}
\vspace{2\baselineskip}

\section{Mostek Wiena}
\subsection{Mostek Wiena $R=10k\Omega$, $C=100nF$}

\begin{figure}[H]
\centering
\includegraphics[width=0.5\textwidth]{img/mostekwiena.png}
\caption{Mostek Wiena}
\end{figure}

Gdzie $R = 10 k\Omega$ i $C = 100 nF$, stąd wiemy ze czestotliwość pseudorezonansowa wynosi:

\begin{equation}
f_{0 \text{ obliczone}} = \frac{1}{2\pi RC} = \frac{1}{2\pi \cdot 10^4 \cdot 10^{-7}} \approx 159 Hz
\end{equation}

\begin{figure}[H]
\centering
\includegraphics[width=0.5\textwidth]{img/charakterystyka copy.png}
\caption{Charakterystyka mostka Wiena}
\end{figure}

\[
f_{0 \text{ zmierzone}} = 181 Hz
\]

\subsection{Mostek Wiena $R=10k\Omega$, $C=22nF$}

Dla $R=10k\Omega$ i $C=22nF$ przeprowadziliśmy analize w taki sam sposób jak dla poprzedniego układu pomiary.

\begin{equation}
f_{0 \text{obliczone}} = \frac{1}{2\pi RC} = \frac{1}{2\pi \cdot 10^4 \cdot 22^{-9}} \approx 589 Hz
\end{equation}

\[
f_{0 \text{ zmierzone}} = 865 Hz
\]




\subsection{Generator RC}

\subsubsection{Mostek Wiena $R=10k\Omega$, $C=100nF$}

\begin{equation}
K = \frac{u_{gen}}{u_H} = \frac{4,87}{1,65} \approx 2,96
\end{equation}

\subsubsection{Mostek Wiena $R=10k\Omega$, $C=22nF$}

\begin{equation}
K = \frac{u_{gen}}{u_H} = \frac{5,03}{1,7} \approx 2,95
\end{equation}


\section{Podsumowanie}


\begin{table}[H]
\centering
\begin{tabular}{|c|cc|cccc|}
\hline
\multirow{2}{*}{Badany układ} & \multicolumn{2}{c|}{Mostek Wiena}                                                        & \multicolumn{4}{c|}{Generator}                                                                                                                       \\ \cline{2-7} 
                              & \multicolumn{1}{c|}{$f_0$ zmierzone [Hz]} & $f_0$ obliczone [Hz] & \multicolumn{1}{c|}{$f_{gen}$ [Hz]} & \multicolumn{1}{c|}{$u_{gen}$ [V]} & \multicolumn{1}{c|}{$u_H$ [V]} & K    \\ \hline
$10k\Omega$ 100nF             & \multicolumn{1}{c|}{181}                              & 159                              & \multicolumn{1}{c|}{181}                        & \multicolumn{1}{c|}{5,03}                      & \multicolumn{1}{c|}{1,7}                   & 2,96 \\ \hline
$10k\Omega$ 22nF              & \multicolumn{1}{c|}{865}                              & 723                              & \multicolumn{1}{c|}{865}                        & \multicolumn{1}{c|}{4,87}                      & \multicolumn{1}{c|}{1,65}                  & 2,95 \\ \hline
\end{tabular}
\end{table}


\section{Wnioski}

\subsection{Porównanie $f_0$ obliczone vs $f_0$ zmierzone}
Wartość częstotliwości pseudorezonansowej mostka Wiena została zmierzona z charakterystyki fazowej jako wartość częstotliwości dla przesunięcia fazowego równego 0 i jako wartość częstotliwości dla wzmocnienia równego $\frac{1}{3}$. W obu przypadkach do porównania zostały użyte wartości czestotliwości pseudorezonansowej odczytane z przesunięcia fazowego równego 0.
Odczytane wartości częstotliwości pseudorezonansowejwykazują pewne rozbieżności z wartościami obliczonymi teoretycznie. Różnice te wynikają głównie z faktu, że punkt przejścia fazy przez 0\degree\ nie pokrywa się idealnie z punktem maksymalnego wzmocnienia, co jest efektem tolerancji elementów RC oraz parametrów pasożytniczych układu.

Zmierzone wartości wzmocnienia ($k \approx 2,96$ oraz $k \approx 2,95$) są zbliżone do teoretycznej wartości $k=3$, co potwierdza spełnienie warunku amplitudy.

\subsection{Generator RC z mostkiem Wiena}
Kluczowym elementem zapewniającym stabilność i jakość generowanego sygnału jest układ automatycznej regulacji wzmocnienia zrealizowany na diodach Zenera.
Przy optymalnym wzmocnieniu ($k \approx 3$) diody Zenera zaczynają przewodzić przy wzroście amplitudy, dynamicznie zmniejszając rezystancję w pętli sprzężenia zwrotnego. Dzięki temu wzmocnienie jest redukowane do poziomu jedności dla pętli zamkniętej, co stabilizuje amplitudę i zapewnia czysty przebieg sinusoidalny.
Przeprowadzone badania potwierdziły, że mostek Wiena w połączeniu z nieliniową pętlą sprzężenia zwrotnego stanowi stabilne i przestrajalne źródło napięcia sinusoidalnego.

\end{document}
