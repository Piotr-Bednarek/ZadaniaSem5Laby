\documentclass[11pt]{article}

\pdfobjcompresslevel=3    % compress PDF objects
\pdfcompresslevel=9       % compress streams more (0..9)
\pdfminorversion=4


\usepackage[utf8]{inputenc} % remove if using XeLaTeX or LuaLaTeX
\usepackage[T1]{fontenc}

% Layout & graphics
\usepackage[a4paper, total={7.5in, 9in}]{geometry}
\usepackage{graphicx}              % images
\usepackage[dvipsnames,table]{xcolor} % single xcolor invocation (keeps table option)

% Math & symbols
\usepackage{amsmath}
\usepackage{amssymb}
\usepackage{gensymb}

% Typography & layout helpers
\usepackage{microtype}
\usepackage{float}                 % [H] placement
\usepackage{caption}
\usepackage{subcaption}            % modern subfigure support (do NOT load subfig)
\usepackage{multirow}
\usepackage{titlesec}

\usepackage{lmodern}
\usepackage{microtype}

\definecolor{xppblue}{RGB}{37, 150, 190}

\begin{document}

\begin{table}[h]
\centering
\scalebox{1.5}{
\begin{tabular}{|ccll|cl|}
\hline
\multicolumn{4}{|c|}{SPRAWOZDANIE Z LABORATORIUM}                                                                                                                            & \multicolumn{2}{c|}{\multirow{2}{*}{\begin{tabular}[c]{@{}c@{}}rok akademicki:\\ 2025/26\end{tabular}}} \\ \cline{1-4}
\multicolumn{4}{|c|}{\textbf{Układy elektroniki użytkowej}}                                                                                                                          & \multicolumn{2}{c|}{}                                                                                   \\ \hline
\multicolumn{4}{|c|}{\multirow{2}{*}{\textit{Źródła prądowe}}}                                                                                           & \multicolumn{2}{c|}{\multirow{2}{*}{czwartek 8:00}}                                                \\
\multicolumn{4}{|c|}{}                                                                                                                                                       & \multicolumn{2}{c|}{}                                                                                   \\ \hline
\multicolumn{1}{|c|}{WARiE, AiR, sem 5}          & \multicolumn{3}{c|}{\multirow{2}{*}{\begin{tabular}[c]{@{}c@{}} \underline{1. Jan Andrzejewski}\\2. Mateusz Banaszak\\\end{tabular}}} & \multicolumn{2}{l|}{\multirow{2}{*}{Punkty:}}                                                           \\ \cline{1-1}
\multicolumn{1}{|c|}{13.11.2025}                   & \multicolumn{3}{c|}{}                                                                                                   & \multicolumn{2}{l|}{}                                                                                   \\ \hline
\end{tabular}
}
\end{table}
\vspace{2\baselineskip}

\section{Źródło prądowe na bazie dwóch tranzystorów}

\begin{figure}[H]
    \centering
    \includegraphics[width=0.4\linewidth]{img/schemattran.png}
\end{figure}

Elementy wykorzystane do budowy układu:
\begin{enumerate}
    \item $R_1=100k\Omega$
    \item $Q_1=Q_2=$BC237
\end{enumerate}

$R_2$ musi być dobrane tak żeby amperomierz wskazywal 0.6mA. Wiemy że na rezystorze $R_2$ spadek musi wynosić 0.6V więc musi on mieć rezystancje $1k\Omega$

W rzeczywistosci dla rezystora $R_2=1k\Omega$ amperomierz wskazywał 0.52mA. Taka rozbieżność może wynikac z tolerancji rezystorów i zurzycia tranzystorów. Zmniejszając $R_2$ do $900\Omega$ 

\subsection{Odpowiedzi na pytania}
\subsubsection*{Jakiego typu są tranzystory Q1 i Q2?}
Tranzystory BC237 są transzystorami NPN
\subsubsection*{W jaki sposób dobrać wartość elementów, by 'zaprogramować' prąd? }
Wiadomo że: $I\propto\frac{1}{R}$ więc zwiekszając wartość rezystora możemy wpłynąć na wartość prądu w całej tej gałęzi.
\subsubsection*{Czy wszystkie trzy diody(czerwona/zielona/niebieska) włączone w miejscu amperomierza się zapalą?}
Tak, dopóki napiecie VDD bedzie wieksze od spadku napiecia na diodzie(czerwona $\approx 1.4V$),spadku na tranzystorze(0.2V) i spadku na rezystorze, w tym przypadku 0.6. Wiec VDD>2.2V
\subsubsection*{Czy wszystkie trzy diody się zapalą, jeśli podłączymy je równolegle?}
Nie, prąd popłynie tylko przez diode z najniższym znamionowym spadkiem napięcia.
\subsubsection*{Czy zmiana połączenia diod z szeregowego na równoległe w podanej sytuacji wiąże się z ryzykiem uszkodzenia elementów}
Nie, jedyne o czym trzeba pamiętać to odpowiednio wysokie napiecie zasilania.
\subsubsection*{W których miejcach pomiar napiecia stałego w układzie ma sens?}

\begin{figure}[H]
    \centering
    \includegraphics[width=0.5\linewidth]{img/schematpunkty.png}
\end{figure}

Jakikolwiek sens: C, B

Napięcie na wyjściu źródła prądowego: C

\subsubsection*{Jaką role w układzie pełni tranzysotr Q2?}
Pełni role stabilizowania spadku napiecia na rownolegle podłączonym do niego rezystorze.

\subsubsection*{Co się stanie, jeśli rezystor R2 zastąpimy zwarciem?}
Przy rezystancji dążącej do 0 prąd bedzie dążyć do nieskonczoności co nie jest możliwe. W rzeczywistosci zwarcie R2 spowodowałoby wejście tranzystora w nasycenie, co ograniczy prąd do wartości wynikającej z charakterystyk elementów oraz rezystancji przewodów. 
\subsubsection*{Czy na rezystorze R2 pojawi się rzeczywiście 0.6V? Dlaczego? Które złącze, w którym tranzystorze powoduje ustabilizowanie się tego spadku napięcia?}
Nie jest to tylko w przybliżeniu 0.6V, ten spadek wynika z spadku napiecia na tranzystorze Q2 baza-emiter.
\subsubsection*{W jakim stanie pracuje tranzystor, gdy układ poprawnie stabilizuje prąd?}
Tranzystor pracuje w stanie aktywnym.

\section{Źródło prądowe sterowane napięciem}

\begin{figure}[H]
    \centering
    \includegraphics[width=0.6\linewidth]{img/schematnap.png}
\end{figure}

\subsubsection*{Wykresy zależności}

\begin{table}[!ht]
    \centering
    \begin{tabular}{|l|l|l|l|}
    \hline
        \$U\_s\$ & \$I\_o\$ & \$U\_c\$ & \$U\_e\$ \\ \hline
        1 & 1,16 & 7,23 & 0,65 \\ \hline
        2 & 2,33 & 5,88 & 0,64 \\ \hline
        3 & 3,50 & 4,66 & 0,65 \\ \hline
        4 & 4,65 & 3,46 & 0,66 \\ \hline
        4,4 & 4,94 & 2,99 & 0,66 \\ \hline
        5 & 5,82 & 2,28 & 0,67 \\ \hline
        6 & 6,99 & 1,23 & 0,68 \\ \hline
        7 & 8,15 & 0,01 & 0,69 \\ \hline
        8 & 8,50 & 0,67 & 0,73 \\ \hline
        9 & 8,50 & 0,68 & 0,73 \\ \hline
        10 & 8,50 & 0,69 & 0,73 \\ \hline
        11 & 8,52 & 0,70 & 0,73 \\ \hline
        12 & 8,33 & 0,70 & 0,73 \\ \hline
    \end{tabular}
\end{table}

\begin{figure}[H]
    \centering
    \includegraphics[width=0.7\linewidth]{img/Io=f(Us).png}
\end{figure}

\begin{figure}[H]
    \centering
    \includegraphics[width=0.7\linewidth]{img/Uc,Ue=f(Us).png}
\end{figure}

\subsubsection*{Transkonduktancja}

Transkonduktancja to współczynnik określający zmiennosc prądu na wyjściu układu, w zależności od napiecia na wejściu. 

\[
g_{m \quad teoretyczna}=\frac{1}{R_0}=\frac{1}{1000}=0.001
\]
Wynik ten nalezy interpretowac jako A/V wiec otrzymaliśmy:
\[
g_m=1\frac{mA}{V}
\]

Transkonduktancje z pomiarów możemy wyznaczyc korzystajac z współczynnika kierunkowego prostej uzyskanej poprzez regresje liniową z naszych pomiarów

\[
g_{m \quad wyznaczone}=1.17 \frac{mA}{V}
\]

Różnica pomiędzy wartością teoretyczną a wyznaczoną eksperymentalnie wynika głównie z uproszczonego modelu teoretycznego. W rzeczywistym układzie na wartość transkonduktancji wpływa nieliniowy charakter złącza baza–emiter, zmiany temperatury oraz nieidealne odwzorowanie liniowości dla wyższych napięć sterujących. Mimo to obie wartości pozostają zgodne rzędu wielkości, co potwierdza poprawność działania układu


\subsubsection*{Maksymalny prąd(diody szeregowo)}
\begin{table}[H]
\centering
\begin{tabular}{|l|l|}
\hline
Diody & I{[}mA{]} \\ \hline
-     & 0,012     \\ \hline
1     & 0,012     \\ \hline
2     & 0,010     \\ \hline
3     & 0,007     \\ \hline
\end{tabular}
\end{table}


\section{Wnioski}

Porównanie wyników teoretycznych i pomiarowych pokazuje zgodność charakteru działania obu źródeł prądowych. Różnice wynikają głównie z rzeczywistych parametrów tranzystorów, w szczególności zmienności napięcia $U_{BE}$, które zależy od temperatury oraz prądu. W układzie sterowanym napięciem obserwujemy wyraźny zakres pracy liniowej oraz obszar nasycenia, co ogranicza maksymalną wartość prądu. Stabilizacja prądu w układzie na bazie tranzystorów potwierdza skuteczność sprzężenia zwrotnego wykorzystującego spadek napięcia baza–emiter. Przeprowadzone pomiary pozwalają ocenić zarówno zalety, jak i ograniczenia prostych źródeł prądowych zbudowanych na tranzystorach bipolarnych



\end{document}