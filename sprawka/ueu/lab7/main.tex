\documentclass[11pt]{article}

\pdfobjcompresslevel=3    % compress PDF objects
\pdfcompresslevel=9       % compress streams more (0..9)
\pdfminorversion=4


\usepackage[utf8]{inputenc} % remove if using XeLaTeX or LuaLaTeX
\usepackage[T1]{fontenc}

% Layout & graphics
\usepackage[a4paper, total={7.5in, 9in}]{geometry}
\usepackage{graphicx}              % images
\usepackage[dvipsnames,table]{xcolor} % single xcolor invocation (keeps table option)

% Math & symbols
\usepackage{amsmath}
\usepackage{amssymb}
\usepackage{gensymb}

% Typography & layout helpers
\usepackage{microtype}
\usepackage{float}                 % [H] placement
\usepackage{caption}
\usepackage{subcaption}            % modern subfigure support (do NOT load subfig)
\usepackage{multirow}
\usepackage{titlesec}

\usepackage{lmodern}
\usepackage{microtype}

\definecolor{xppblue}{RGB}{37, 150, 190}

\begin{document}

\begin{table}[h]
\centering
\scalebox{1.5}{
\begin{tabular}{|ccll|cl|}
\hline
\multicolumn{4}{|c|}{SPRAWOZDANIE Z LABORATORIUM}                                                                                                                            & \multicolumn{2}{c|}{\multirow{2}{*}{\begin{tabular}[c]{@{}c@{}}rok akademicki:\\ 2025/26\end{tabular}}} \\ \cline{1-4}
\multicolumn{4}{|c|}{\textbf{Układy elektroniki użytkowej}}                                                                                                                          & \multicolumn{2}{c|}{}                                                                                   \\ \hline
\multicolumn{4}{|c|}{\multirow{2}{*}{\textit{Źródła prądowe}}}                                                                                           & \multicolumn{2}{c|}{\multirow{2}{*}{czwartek 8:00}}                                                \\
\multicolumn{4}{|c|}{}                                                                                                                                                       & \multicolumn{2}{c|}{}                                                                                   \\ \hline
\multicolumn{1}{|c|}{WARiE, AiR, sem 5}          & \multicolumn{3}{c|}{\multirow{2}{*}{\begin{tabular}[c]{@{}c@{}} \underline{1. Jan Andrzejewski}\\2. Mateusz Banaszak\\\end{tabular}}} & \multicolumn{2}{l|}{\multirow{2}{*}{Punkty:}}                                                           \\ \cline{1-1}
\multicolumn{1}{|c|}{13.11.2025}                   & \multicolumn{3}{c|}{}                                                                                                   & \multicolumn{2}{l|}{}                                                                                   \\ \hline
\end{tabular}
}
\end{table}
\vspace{2\baselineskip}

\section{Źródło prądowe na bazie dwóch tranzystorów}

\begin{figure}[H]
    \centering
    \includegraphics[width=0.4\linewidth]{img/schemattran.png}
\end{figure}

Elementy wykorzystane do budowy układu:
\begin{enumerate}
    \item $R_1=100k\Omega$
    \item $Q_1=Q_2=$BC237
\end{enumerate}

$R_2$ musi być dobrane tak żeby amperomierz wskazywal 0.6mA. Wiemy że na rezystorze $R_2$ spadek musi wynosić 0.6V więc musi on mieć rezystancje $1k\Omega$

W rzeczywistosci dla rezystora $R_2=1k\Omega$ amperomierz wskazywał 0.52mA. Taka rozbieżność może wynikac z tolerancji rezystorów i zurzycia tranzystorów. Zmniejszając $R_2$ do $900\Omega$ 

\subsection{Odpowiedzi na pytania}
\subsubsection*{Jakiego typu są tranzystory Q1 i Q2?}
Tranzystory BC237 są transzystorami NPN
\subsubsection*{W jaki sposób dobrać wartość elementów, by 'zaprogramować' prąd? }
Wiadomo że: $I\propto\frac{1}{R}$ więc zwiekszając wartość rezystora możemy wpłynąć na wartość prądu w całej tej gałęzi.
\subsubsection*{Czy wszystkie trzy diody(czerwona/zielona/niebieska) włączone w miejscu amperomierza się zapalą?}
Tak, dopóki napiecie VDD bedzie wieksze od spadku napiecia na diodzie(czerwona $\approx 1.4V$),spadku na tranzystorze(0.2V) i spadku na rezystorze, w tym przypadku 0.6. Wiec VDD>2.2V
\subsubsection*{Czy wszystkie trzy diody się zapalą, jeśli podłączymy je równolegle?}
Nie, prąd popłynie tylko przez diode z najniższym znamionowym spadkiem napięcia.
\subsubsection*{Czy zmiana połączenia diod z szeregowego na równoległe w podanej sytuacji wiąże się z ryzykiem uszkodzenia elementów}|
Nie, jedyne o czym trzeba pamiętać to odpowiednio wysokie napiecie zasilania.




\end{document}