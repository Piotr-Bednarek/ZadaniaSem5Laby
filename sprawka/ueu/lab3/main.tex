\documentclass[11pt]{article}

\pdfobjcompresslevel=3    % compress PDF objects
\pdfcompresslevel=9       % compress streams more (0..9)
\pdfminorversion=4


\usepackage[utf8]{inputenc} % remove if using XeLaTeX or LuaLaTeX
\usepackage[T1]{fontenc}

% Layout & graphics
\usepackage[a4paper, total={7.5in, 9in}]{geometry}
\usepackage{graphicx}              % images
\usepackage[dvipsnames,table]{xcolor} % single xcolor invocation (keeps table option)

% Math & symbols
\usepackage{amsmath}
\usepackage{amssymb}
\usepackage{gensymb}

% Typography & layout helpers
\usepackage{microtype}
\usepackage{float}                 % [H] placement
\usepackage{caption}
\usepackage{subcaption}            % modern subfigure support (do NOT load subfig)
\usepackage{multirow}
\usepackage{titlesec}

\usepackage{lmodern}
\usepackage{microtype}

\definecolor{xppblue}{RGB}{37, 150, 190}

\begin{document}

\begin{table}[h]
\centering
\scalebox{1.5}{
\begin{tabular}{|ccll|cl|}
\hline
\multicolumn{4}{|c|}{SPRAWOZDANIE Z LABORATORIUM}                                                                                                                            & \multicolumn{2}{c|}{\multirow{2}{*}{\begin{tabular}[c]{@{}c@{}}rok akademicki:\\ 2024/25\end{tabular}}} \\ \cline{1-4}
\multicolumn{4}{|c|}{\textbf{Układy elektroniki użytkowej}}                                                                                                                          & \multicolumn{2}{c|}{}                                                                                   \\ \hline
\multicolumn{4}{|c|}{\multirow{2}{*}{\textit{Aktywne układy liniowe}}}                                                                                           & \multicolumn{2}{c|}{\multirow{2}{*}{czwartek 8:00}}                                                \\
\multicolumn{4}{|c|}{}                                                                                                                                                       & \multicolumn{2}{c|}{}                                                                                   \\ \hline
\multicolumn{1}{|c|}{WARiE, AiR, sem 5}          & \multicolumn{3}{c|}{\multirow{2}{*}{\begin{tabular}[c]{@{}c@{}} \underline{1. Jan Andrzejewski}\\2. Mateusz Banaszak\\\end{tabular}}} & \multicolumn{2}{l|}{\multirow{2}{*}{Punkty:}}                                                           \\ \cline{1-1}
\multicolumn{1}{|c|}{16.10.2025}                   & \multicolumn{3}{c|}{}                                                                                                   & \multicolumn{2}{l|}{}                                                                                   \\ \hline
\end{tabular}
}
\end{table}
\vspace{2\baselineskip}

\section{Cel i zakres ćwiczenia}
Celem ćwiczenia jest zapoznanie się z działaniem prostych układów liniowych zrealizowanych
na wzmacniaczu operacyjnym, a także ze specyfiką pomiarów w takich układach. W pomiarach
zostaną wykorzystane wirtualne przyrządy pomiarowe wygenerowane w programie LabView oraz
urządzenie ELVIS II.

\section{Sumator}

\begin{figure}[H]
    \centering
    \includegraphics[width=0.5\linewidth]{img/image.png}
    \caption{Schemat układu sumatora}
    \label{fig:placeholder}
\end{figure}

Napięcie $U_2$ można okreslić jako:
\[
U_3=-R_3\bigg( \frac{U_1}{R_1} +\frac{U_2}{R_2}\bigg)
\]
lub jako:
\[
U_3=-\frac{R_3}{R_1}U_1-\frac{R_3}{R_2}U_2
\]
gdzie:
\[
R_1=2k\Omega
\] 
\[
R_2=1k\Omega
\] 
\[
R_3=2k\Omega
\] 
wiec:
\[
-U_3=U_1+2\cdot U_2 
\]

sygnał wynikowy bedzie przeciwną w fazie sumą sygnału $U_1$ i dwukrotności $U_2$

\begin{figure}[H]
    \centering
    \includegraphics[width=0.8\linewidth]{img/sumator.jpg}
    \caption{Przebiegi $U_1$, $U_2$ i $U_3$ nastawy 1 sumator}
    \label{fig:placeholder}
\end{figure}

\begin{figure}[H]
    \centering
    \includegraphics[width=0.8\linewidth]{img/sumator3.jpg}
    \caption{Przebiegi $U_1$, $U_2$ i $U_3$ nastawy 2 sumator}
    \label{fig:placeholder}
\end{figure}

\begin{table}[H]
\centering
\begin{tabular}{|l|l|l|l|l|l|l|l|l|}
\hline
 &$U_1$ [V] & $U_1$ RMS[V] & $U_2$[V] & $U_2$ RMS[V] & wzmocnienie & $U_3$[V] & $U_3$ RMS[V] obl.& $U_3$ RMS[V] zmie. \\ \hline
1&   4  &      2,83    &  1    &      0,707    &    $U_1 +2\cdot U_2$     & 6     &   4,24     &  4,26 \\ \hline
2&   3  &      2,12    &   2   &      1,41    &      $U_1 -2\cdot U_2$     &  -1    &   -0,70   &  0,649  \\ \hline
\end{tabular}
\end{table}

Obecność plusa bądź minusa zależy od faz sygnałów wchodzących do sumatora.Jeśli są zgodne w fazie 
pojawia sie plus, jeśli przeciwne to minus. Wartości obliczone pokrywają się z wartościami pomiarów 


\section{Wzmacniacz różnicowy}

\begin{figure}[H]
    \centering
    \includegraphics[width=0.5\linewidth]{img/roznicowy.png}
    \caption{Schemat połączeń w wzmacniaczu różnicowym}
    \label{fig:placeholder}
\end{figure}

Wzmocnienie wzmacniacza różnicowego wyraża sie jako:
\[
U_3=\frac{R_B}{R_A}(U_2-U_1)
\]
gdzie:
\[
R_A=R_1=R_2
\]
\[
R_B=R_3=R_4
\]
\[
R_A=1k\Omega
\]
\[
R_B=2k\Omega
\]

więc sygnał na wyjściu wzmacniacza bedzie dwukrotnym 
wzmocnieniem róznicy sygnałów wejściowych 

\begin{figure}[H]
    \centering
    \includegraphics[width=0.8\linewidth]{img/roznicowy1.jpg}
    \caption{Przebiegi $U_1$, $U_2$ i $U_3$ nastawy 1 wzmacniacz różnicowy}
    \label{fig:placeholder}
\end{figure}

\begin{figure}[H]
    \centering
    \includegraphics[width=0.8\linewidth]{img/roznicowy2.jpg}
    \caption{Przebiegi $U_1$, $U_2$ i $U_3$ nastawy 2 wzmacniacz różnicowy}
    \label{fig:placeholder}
\end{figure}

\begin{table}[H]
\centering
\begin{tabular}{|l|l|l|l|l|l|l|l|l|}
\hline
 &$U_1$ [V] & $U_1$ RMS[V] & $U_2$[V] & $U_2$ RMS[V] & wzmocnienie & $U_3$[V] & $U_3$ RMS[V] obl.& $U_3$ RMS[V] zmie. \\ \hline
1&   4  &      2,83    &  4    &      2,83    &    $2(U_1 - U_2)$     & 0     &   0     &  0,0049 \\ \hline
2&   1  &      0,707    &   5   &      3,53    &      $2(U_1 - U_2)$     &  8    &   5,65   &  5,6  \\ \hline
\end{tabular}
\end{table}

Obecność plusa bądź minusa we wzmocnieniu zależy od tego czy sygnały są 
zgodne czy przeciwne w fazie. sygnały zgodne w fazie to minus, przeciwne plus.

\section{Pomiar CMRR}

Współczynnik CMRR wzmacniacza róznicowego określa zdolność wzmacniacza do tłumenia sygnałów wspólnych. Czyli zdolność wzmacniacza do tłumenia 
sygnałów o takiej samej amplitudzie i fazie na obu wejściach wzmacniacza.
\[
CMRR=20log\frac{U_{wy1}}{U_{wy2}}
\]

\subsection{200Hz}

\begin{figure}[H]
    \centering
    \includegraphics[width=0.8\linewidth]{img/cmrr1200.jpg}
    \caption{Przebieg $U_{wy1}$ dla f=200Hz}
    \label{fig:placeholder}
\end{figure}

\begin{figure}[H]
    \centering
    \includegraphics[width=0.8\linewidth]{img/cmrr2200.jpg}
    \caption{Przebieg $U_{wy2}$ dla f=200Hz}
    \label{fig:placeholder}
\end{figure}

\[
U_{wy1}=6,38V
\]
\[
U_{wy2}=0,0352V
\]
\[
CMRR_{2kHz}=20log\frac{6,38}{0,0352}=45,165dB
\]

\subsection{2kHz}

\begin{figure}[H]
    \centering
    \includegraphics[width=0.8\linewidth]{img/cmrr1pom.jpg}
    \caption{Przebieg $U_{wy1}$ dla f=2kHz}
    \label{fig:placeholder}
\end{figure}

\begin{figure}[H]
    \centering
    \includegraphics[width=0.8\linewidth]{img/cmrr2pomnasz.jpg}
    \caption{Przebieg $U_{wy2}$ dla f=2kHz}
    \label{fig:placeholder}
\end{figure}

\[
U_{wy1}=3,15V
\]
\[
U_{wy2}=0,00903V
\]
\[
CMRR_{2kHz}=20log\frac{3,15}{0,00903}=50,852dB
\]

Różnica pomiędzy wartościami tłumienia wynika z skończonego pasma przenoszenia wzmacniacza  


\section{Wnioski}
 Przeprowadziliśmy pomiary układów i porównaliśmy
 je z przewidywaniami teoretycznymi. Sumator odwracający zachowywał się zgodnie z założeniami
,napięcie wyjściowe jest odwrotnością fazową ważonej sumy sygnałów wejściowych, wagi wejsc zaleza 
od dzielnika napięć postającego na rezystorze w torze sprzężenia zwrotnego i rezystorach na liniach wejść, wartości obliczone zgadzają się z pomiarami. Wzmacniacz różnicowy wykazał
   oczekiwane wzmocnienie różnicy napięć i przy zrównanych sygnałach wejściowych daje niemal 
   zerowe wyjście. Zmierzony współczynnik CMRR, świadczy o dobrym, lecz nie idealnym 
   tłumieniu sygnałów wspólnych, ograniczenia wynikają z niedoskonałości elementów i wzmacniacza. Główne źródła
    niepewności to tolerancje rezystorów i ograniczenia pasma.






\end{document}