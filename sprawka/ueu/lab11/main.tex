\documentclass[11pt]{article}

\pdfobjcompresslevel=3    % compress PDF objects
\pdfcompresslevel=9       % compress streams more (0..9)
\pdfminorversion=4


\usepackage[utf8]{inputenc} % remove if using XeLaTeX or LuaLaTeX
\usepackage[T1]{fontenc}

% Layout & graphics
\usepackage[a4paper, total={7.5in, 9in}]{geometry}
\usepackage{graphicx}              % images
\usepackage[dvipsnames,table]{xcolor} % single xcolor invocation (keeps table option)

% Math & symbols
\usepackage{amsmath}
\usepackage{amssymb}
\usepackage{gensymb}

% Typography & layout helpers
\usepackage{microtype}
\usepackage{float}                 % [H] placement
\usepackage{caption}
\usepackage{subcaption}            % modern subfigure support (do NOT load subfig)
\usepackage{multirow}
\usepackage{titlesec}

\usepackage{lmodern}
\usepackage{microtype}

\definecolor{xppblue}{RGB}{37, 150, 190}

\begin{document}

\begin{table}[h]
\centering
\scalebox{1.5}{
\begin{tabular}{|ccll|cl|}
\hline
\multicolumn{4}{|c|}{SPRAWOZDANIE Z LABORATORIUM}                                                                                                                            & \multicolumn{2}{c|}{\multirow{2}{*}{\begin{tabular}[c]{@{}c@{}}rok akademicki:\\ 2025/26\end{tabular}}} \\ \cline{1-4}
\multicolumn{4}{|c|}{\textbf{Układy elektroniki użytkowej}}                                                                                                                          & \multicolumn{2}{c|}{}                                                                                   \\ \hline
\multicolumn{4}{|c|}{\multirow{2}{*}{\textit{UKŁAD CZASOWY NE555}}}                                                                                           & \multicolumn{2}{c|}{\multirow{2}{*}{czwartek 8:00}}                                                \\
\multicolumn{4}{|c|}{}                                                                                                                                                       & \multicolumn{2}{c|}{}                                                                                   \\ \hline
\multicolumn{1}{|c|}{WARiE, AiR, sem 5}          & \multicolumn{3}{c|}{\multirow{2}{*}{\begin{tabular}[c]{@{}c@{}} \underline{1. Jan Andrzejewski}\\2. Mateusz Banaszak\\\end{tabular}}} & \multicolumn{2}{l|}{\multirow{2}{*}{Punkty:}}                                                           \\ \cline{1-1}
\multicolumn{1}{|c|}{11.12.2025}                   & \multicolumn{3}{c|}{}                                                                                                   & \multicolumn{2}{l|}{}                                                                                   \\ \hline
\end{tabular}
}
\end{table}
\vspace{2\baselineskip}

\section{Zasada działania generatora impulsu}



\section{Generator pojedynczego impulsu}

\begin{figure}[H]
    \centering
    \includegraphics[width=0.5\linewidth]{img/pojedynczy.png}
    \caption{Schemat układu generatora pojedynczego impulsu}
\end{figure}

\subsection*{Pierwszy zestaw}

Dobierajac parametry R i C jestesmy w stanie określić stała czasowa impulsu. Wybraliśmy $R=100k\Omega$ i $C=47\mu F$ otrzymamy stałą czsowa równą:
\[
\tau_{imp}=RCln(3)
\]
\[
\tau_{imp}=5,1635 s 
\]


\begin{table}[H]
    \centering
    \begin{tabular}{|l|l|}
    \hline
         lp. & Czas [s] \\ \hline
        1 & 6,5 \\ \hline
        2 & 6,51 \\ \hline
        3 & 6,51 \\ \hline
        4 & 6,51 \\ \hline
        5 & 6,51 \\ \hline
    \end{tabular}
\end{table}


Z dużą pewnościa jesteśmy w stanie stwierdzić że układ w rzeczywistości miał stałą czasową $\tau_{imp}=6,51s$ czyli wiecej niż otrzymaliśmy teoretycznie,


\subsection*{Drugi zestaw}

W tym przypadku wartości R i C wynosiły odpowiednio $100k\Omega$ i $10\mu F$ więc:
\[
\tau_{imp}=1,098 s
\]

\begin{table}[H]
    \centering
    \begin{tabular}{|l|l|}
    \hline
         lp. & Czas [s] \\ \hline
        1 & 1,16 \\ \hline
        2 & 1,15 \\ \hline
        3 & 1,15 \\ \hline
        4 & 1,15 \\ \hline
        5 & 1,15 \\ \hline
    \end{tabular}
\end{table}

\subsection*{Impuls odpowiadający nacisnięciu guzika}

Układ generatora impulsu można wykorzystac jako zabezpieczenie przed drganiami styków guzików poprzez generowanie odpowiednio krótkiego impulsu. Dobraliśmy R i C układu w taki sposób aby uzyskac zadowalająco krótki impuls. Odpowiednio $100k\Omega$ i $2,2\mu F$.
\[
\tau_{imp}=0,2417 s 
\]

\begin{table}[H]
    \centering
    \begin{tabular}{|l|l|}
    \hline
         lp. & Czas [s] \\ \hline
        1 & 0,23 \\ \hline
        2 & 0,22 \\ \hline
        3 & 0,23 \\ \hline
        4 & 0,22 \\ \hline
        5 & 0,23 \\ \hline
    \end{tabular}
\end{table}

\section{Generator relaksacyjny}

\begin{figure}[H]
    \centering
    \includegraphics[width=0.5\linewidth]{img/relaksacyjny.png}
    \caption{Schemat układu generatora relaksacyjnego}
\end{figure}

\begin{figure}[H]
    \centering
    \includegraphics[width=0.5\linewidth]{img/Bez nazwy.jpg}
    \caption{Zestaw instrumentów wykorzystanych w doświadczeniu}
\end{figure}

Teoretyczna czestotliowość generowania impulsów:
\[
f=\frac{1}{(R_1+P+2R_2)Cln(2)}
\]

Gdzie $R_1$ i $R_2$ to odpowiednio $5k\Omega$ i $1k\Omega$ i $C=1\mu F$

\[
f_max=\frac{1}{7 \cdot 10^3 \cdot 10^{-6} ln(2)}=206Hz
\]
\[
f_min=\frac{1}{17\cdot 10^3 \cdot 10^{-6} ln(2)}=84Hz
\]

\begin{table}[H]
    \centering
\begin{tabular}{|l|l|l|l|l|l|}
\hline
Potencjometr & R{[}$\Omega${]} & Vmin{[}V{]} & Vmax{[}V{]} & f{[}Hz{]} & $f_{pomiar}${[}Hz{]} \\ \hline
0            & 1,543           & 1,75        & 3,21        & 207       & 206               \\ \hline
10           & 12,6k           & 1,66        & 3,27        & 124       & 84                \\ \hline
\end{tabular}
\end{table}


Pomiary powtórzono dla $C=100nF$

\[
f_max=\frac{1}{7 \cdot 10^3 \cdot 10^{-7} ln(2)}=2061Hz
\]
\[
f_min=\frac{1}{17\cdot 10^3 \cdot 10^{-7} ln(2)}=848,64Hz
\]

\begin{table}[!ht]
    \centering
    \begin{tabular}{|l|l|l|l|l|l|}
    \hline
        Potencjometr  & R[$\backslash$Omega] & Vmin[V] & Vmax[V] & f[Hz] & $f_{pomiar}${[}Hz{]}\\ \hline
        0 & 1,543 & 1,68 & 3,25 & 2073& 2061\\ \hline
        10 & 12,6k & 1,64 & 3,26 & 1248 & 1248\\ \hline
    \end{tabular}
\end{table}


\section{Pytania}

\subsection{Generator impulsu (monostabilny)}

\subsubsection*{Co się stanie, jeśli chwilę po wyzwoleniu pojawi się kolejny impuls? Czy układ się zresetuje i zacznie liczyć od początku, czy zignoruje go?}
Układ jest niewyzwalalny w czasie trwania impulsu wyjściowego. Kolejny impuls wyzwalający (opadające zbocze na wejściu TRIGGER) zostanie zignorowany, a czas trwania impulsu wyjściowego nie ulegnie zmianie.

\subsubsection*{Co się stanie, jeśli przytrzymamy przycisk Wyzwalanie (Direct) dłużej niż wynosi czas (ze stałej RC)?}
Jeśli wejście wyzwalające (TRIGGER) będzie utrzymywane w stanie niskim dłużej niż wynosi czas impulsu wynikający ze stałej czasowej $RC$, wyjście pozostanie w stanie wysokim tak długo, jak wciśnięty jest przycisk. Dopiero po zwolnieniu przycisku (powrót wejścia w stan wysoki) wyjście przejdzie w stan niski (ponieważ kondensator jest już naładowany powyżej $2/3 V_{CC}$).

\subsubsection*{Do jakiego napięcia naładuje się kondensator?}
Kondensator ładuje się do napięcia $2/3 V_{CC}$ (napięcie progowe komparatora THRESHOLD).

\subsubsection*{Do jakiego napięcia rozładuje się kondensator?}
W stanie spoczynku kondensator jest rozładowany do 0 V (przez tranzystor rozładowujący DISCHARGE).

\subsection{Generator relaksacyjny (astabilny)}

\subsubsection*{Przez jaki rezystor kondensator się ładuje?}
Kondensator ładuje się przez szeregowe połączenie rezystorów $R_1$, potencjometru $P$ oraz rezystora $R_2$ (droga od $V_{CC}$ do kondensatora).

\subsubsection*{Przez jaki rezystor(y) kondensator się rozładowuje?}
Kondensator rozładowuje się przez rezystor $R_2$ (do pinu 7 DISCHARGE).

\subsubsection*{Dlaczego we wzorze mamy $2 \cdot R_2$ a nie $1 \cdot R_2$?}
Okres drgań $T$ składa się z czasu ładowania ($t_{high}$) i czasu rozładowania ($t_{low}$).
$t_{high} \sim (R_1 + P + R_2) \cdot C$
$t_{low} \sim R_2 \cdot C$
Suma czasów daje: $T \sim (R_1 + P + 2R_2) \cdot C$. Rezystancja $R_2$ bierze udział w obu fazach cyklu, dlatego we wzorze na częstotliwość występuje podwojona.

\subsubsection*{Czy zmiana nastawy potencjometru wpływa na długość trwania stanu wysokiego?}
Tak, ponieważ potencjometr znajduje się w gałęzi ładowania kondensatora.

\subsubsection*{Czy zmiana nastawy potencjometru wpływa na długość trwania stanu niskiego?}
Nie, ponieważ rozładowanie odbywa się tylko przez rezystor $R_2$, a potencjometr (w przyjętej konfiguracji $R_1+P$) jest pomijany w tej fazie.

\subsubsection*{Do jakiego napięcia naładuje się kondensator?}
Do $2/3 V_{CC}$.

\subsubsection*{Do jakiego napięcia rozładuje się kondensator?}
Do $1/3 V_{CC}$.

\subsubsection*{Czy można zmienić napięcie, do jakiego ładuje się kondensator? Jak? Czy wpłynie to na napięcie do którego rozładowuje się kondensator?}
Tak, można to zmienić podając zewnętrzne napięcie na wejście sterujące (CONTROL VOLTAGE, pin 5). Zmieni to górny próg przełączania (z domyślnego $2/3 V_{CC}$ na $V_{CTL}$). Dolny próg przełączania zmieni się proporcjonalnie i wyniesie $1/2 V_{CTL}$ (wewnętrzny dzielnik napięcia dzieli je w stosunku 1:1 w tym punkcie). Zatem zmiana napięcia sterującego wpływa na oba progi.



\section{Wnioski}

W ćwiczeniu zbadano dwa podstawowe układy pracy timera NE555: generator monostabilny oraz generator astabilny (relaksacyjny).

\subsection*{Generator monostabilny}
Pomiary czasu trwania impulsu wykazały zbieżność z wartościami teoretycznymi, jednak zauważono istotne różnice w przypadku kondensatora o dużej pojemności ($47\mu F$). Zmierzony czas ($6,51s$) był wyraźnie dłuższy od obliczonego ($5,16s$). Różnica ta wynika najprawdopodobniej z dużej tolerancji wykonania kondensatorów elektrolitycznych (często sięgającej $\pm 20\%$) oraz prądu upływu, który w przypadku długich czasów ładowania zaczyna odgrywać rolę. Dla mniejszych pojemności ($10\mu F$ i $2,2\mu F$) błędy były znacznie mniejsze, co potwierdza tezę o wpływie jakości kondensatora na precyzję układu.
Potwierdzono również przydatność układu do eliminacji drgań styków – układ generował stabilny impuls o zadanej długości mimo potencjalnych zakłóceń na wejściu wyzwalającym.

\subsection*{Generator relaksacyjny}
W konfiguracji astabilnej uzyskano bardzo wysoką zgodność częstotliwości zmierzonych z obliczonymi (np. $206Hz$ vs $207Hz$). Układ poprawnie reagował na zmiany rezystancji w pętli ładowania (potencjometr), co pozwalało na regulację częstotliwości i wypełnienia przebiegu.
Analiza układu potwierdziła, że częstotliwość pracy zależy od sumy rezystancji w procesie ładowania i tylko od $R_2$ w procesie rozładowania, co powoduje, że wypełnienie impulsu jest zawsze większe od 50\% w podstawowej konfiguracji.

\subsection*{Podsumowanie}
Układ NE555 jest uniwersalnym i prostym w użyciu układem, a zgodność jego parametrów z teorią jest wysoka, pod warunkiem zastosowania elementów biernych o odpowiedniej precyzji.

\end{document}