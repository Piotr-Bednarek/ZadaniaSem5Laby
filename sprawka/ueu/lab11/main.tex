\documentclass[11pt]{article}

\pdfobjcompresslevel=3    % compress PDF objects
\pdfcompresslevel=9       % compress streams more (0..9)
\pdfminorversion=4


\usepackage[utf8]{inputenc} % remove if using XeLaTeX or LuaLaTeX
\usepackage[T1]{fontenc}

% Layout & graphics
\usepackage[a4paper, total={7.5in, 9in}]{geometry}
\usepackage{graphicx}              % images
\usepackage[dvipsnames,table]{xcolor} % single xcolor invocation (keeps table option)

% Math & symbols
\usepackage{amsmath}
\usepackage{amssymb}
\usepackage{gensymb}

% Typography & layout helpers
\usepackage{microtype}
\usepackage{float}                 % [H] placement
\usepackage{caption}
\usepackage{subcaption}            % modern subfigure support (do NOT load subfig)
\usepackage{multirow}
\usepackage{titlesec}

\usepackage{lmodern}
\usepackage{microtype}

\definecolor{xppblue}{RGB}{37, 150, 190}

\begin{document}

\begin{table}[h]
\centering
\scalebox{1.5}{
\begin{tabular}{|ccll|cl|}
\hline
\multicolumn{4}{|c|}{SPRAWOZDANIE Z LABORATORIUM}                                                                                                                            & \multicolumn{2}{c|}{\multirow{2}{*}{\begin{tabular}[c]{@{}c@{}}rok akademicki:\\ 2025/26\end{tabular}}} \\ \cline{1-4}
\multicolumn{4}{|c|}{\textbf{Układy elektroniki użytkowej}}                                                                                                                          & \multicolumn{2}{c|}{}                                                                                   \\ \hline
\multicolumn{4}{|c|}{\multirow{2}{*}{\textit{UKŁAD CZASOWY NE555}}}                                                                                           & \multicolumn{2}{c|}{\multirow{2}{*}{czwartek 8:00}}                                                \\
\multicolumn{4}{|c|}{}                                                                                                                                                       & \multicolumn{2}{c|}{}                                                                                   \\ \hline
\multicolumn{1}{|c|}{WARiE, AiR, sem 5}          & \multicolumn{3}{c|}{\multirow{2}{*}{\begin{tabular}[c]{@{}c@{}} \underline{1. Jan Andrzejewski}\\2. Mateusz Banaszak\\\end{tabular}}} & \multicolumn{2}{l|}{\multirow{2}{*}{Punkty:}}                                                           \\ \cline{1-1}
\multicolumn{1}{|c|}{11.12.2025}                   & \multicolumn{3}{c|}{}                                                                                                   & \multicolumn{2}{l|}{}                                                                                   \\ \hline
\end{tabular}
}
\end{table}
\vspace{2\baselineskip}

\section{Zasada działania generatora impulsu}



\section{Generator pojedynczego impulsu}

\begin{figure}[H]
    \centering
    \includegraphics[width=0.5\linewidth]{img/pojedynczy.png}
    \caption{Schemat układu generatora pojedynczego impulsu}
\end{figure}

\subsection*{Pierwszy zestaw}

Dobierajac parametry R i C jestesmy w stanie określić stała czasowa impulsu. Wybraliśmy $R=100k\Omega$ i $C=47\mu F$ otrzymamy stałą czsowa równą:
\[
\tau_{imp}=RCln(3)
\]
\[
\tau_{imp}=5,1635 s 
\]


\begin{table}[H]
    \centering
    \begin{tabular}{|l|l|}
    \hline
         lp. & Czas [s] \\ \hline
        1 & 6,5 \\ \hline
        2 & 6,51 \\ \hline
        3 & 6,51 \\ \hline
        4 & 6,51 \\ \hline
        5 & 6,51 \\ \hline
    \end{tabular}
\end{table}


Z dużą pewnościa jesteśmy w stanie stwierdzić że układ w rzeczywistości miał stałą czasową $\tau_{imp}=6,51s$ czyli wiecej niż otrzymaliśmy teoretycznie,


\subsection*{Drugi zestaw}

W tym przypadku wartości R i C wynosiły odpowiednio $100k\Omega$ i $10\mu F$ więc:
\[
\tau_{imp}=1,098 s
\]

\begin{table}[H]
    \centering
    \begin{tabular}{|l|l|}
    \hline
         lp. & Czas [s] \\ \hline
        1 & 1,16 \\ \hline
        2 & 1,15 \\ \hline
        3 & 1,15 \\ \hline
        4 & 1,15 \\ \hline
        5 & 1,15 \\ \hline
    \end{tabular}
\end{table}

\subsection*{Impuls odpowiadający nacisnięciu guzika}

Układ generatora impulsu można wykorzystac jako zabezpieczenie przed drganiami styków guzików poprzez generowanie odpowiednio krótkiego impulsu. Dobraliśmy R i C układu w taki sposób aby uzyskac zadowalająco krótki impuls. Odpowiednio $100k\Omega$ i $2,2\mu F$.
\[
\tau_{imp}=0,2417 s 
\]

\begin{table}[H]
    \centering
    \begin{tabular}{|l|l|}
    \hline
         lp. & Czas [s] \\ \hline
        1 & 0,23 \\ \hline
        2 & 0,22 \\ \hline
        3 & 0,23 \\ \hline
        4 & 0,22 \\ \hline
        5 & 0,23 \\ \hline
    \end{tabular}
\end{table}

\section{Generator relaksacyjny}

\begin{figure}[H]
    \centering
    \includegraphics[width=0.5\linewidth]{img/relaksacyjny.png}
    \caption{Schemat układu generatora relaksacyjnego}
\end{figure}

\section{Wnioski}

\end{document}
