\documentclass[11pt]{article}

\pdfobjcompresslevel=3    % compress PDF objects
\pdfcompresslevel=9       % compress streams more (0..9)
\pdfminorversion=4


\usepackage[utf8]{inputenc} % remove if using XeLaTeX or LuaLaTeX
\usepackage[T1]{fontenc}

% Layout & graphics
\usepackage[a4paper, total={7.5in, 9in}]{geometry}
\usepackage{graphicx}              % images
\usepackage[dvipsnames,table]{xcolor} % single xcolor invocation (keeps table option)

% Math & symbols
\usepackage{amsmath}
\usepackage{amssymb}
\usepackage{gensymb}

% Typography & layout helpers
\usepackage{microtype}
\usepackage{float}                 % [H] placement
\usepackage{caption}
\usepackage{subcaption}            % modern subfigure support (do NOT load subfig)
\usepackage{multirow}
\usepackage{titlesec}

\usepackage{lmodern}
\usepackage{microtype}

\definecolor{xppblue}{RGB}{37, 150, 190}

\begin{document}

\begin{table}[h]
\centering
\scalebox{1.5}{
\begin{tabular}{|ccll|cl|}
\hline
\multicolumn{4}{|c|}{SPRAWOZDANIE Z LABORATORIUM}                                                                                                                            & \multicolumn{2}{c|}{\multirow{2}{*}{\begin{tabular}[c]{@{}c@{}}rok akademicki:\\ 2025/26\end{tabular}}} \\ \cline{1-4}
\multicolumn{4}{|c|}{\textbf{Układy elektroniki użytkowej}}                                                                                                                          & \multicolumn{2}{c|}{}                                                                                   \\ \hline
\multicolumn{4}{|c|}{\multirow{2}{*}{\textit{Filtry Aktywne}}}                                                                                           & \multicolumn{2}{c|}{\multirow{2}{*}{czwartek 8:00}}                                                \\
\multicolumn{4}{|c|}{}                                                                                                                                                       & \multicolumn{2}{c|}{}                                                                                   \\ \hline
\multicolumn{1}{|c|}{WARiE, AiR, sem 5}          & \multicolumn{3}{c|}{\multirow{2}{*}{\begin{tabular}[c]{@{}c@{}} \underline{1. Jan Andrzejewski}\\2. Mateusz Banaszak\\\end{tabular}}} & \multicolumn{2}{l|}{\multirow{2}{*}{Punkty:}}                                                           \\ \cline{1-1}
\multicolumn{1}{|c|}{30.10.2025}                   & \multicolumn{3}{c|}{}                                                                                                   & \multicolumn{2}{l|}{}                                                                                   \\ \hline
\end{tabular}
}
\end{table}
\vspace{2\baselineskip}

\section{Cel i zakres ćwiczenia}
Celem ćwiczenia jest zapoznanie się z działaniem prostych filtrów aktywnych zrealizowanych
z wykorzystaniem elementów RC i wzmacniacza operacyjnego, a także ze specyfiką pomiarów
filtrów. W ćwiczeniu wykorzystano wirtualne przyrządy pomiarowe opracowane w środowisku
LabVIEW oraz urządzenie ELVIS II.

\section{Filtr dolnoprzepustowy}

\begin{figure}[H]
    \centering
    \includegraphics[width=0.5\linewidth]{img/dolnoschemat.png}
    \caption{aktywny dolnoprzepustowy filtr RC}
\end{figure}

\[
R_1=R_2=R=20k\Omega \qquad C_1=22nF \qquad C_2=44nF
\]

Znajac wartosci elementów z jakich składa sie filtr możemy obliczyć jego częstotliwość graniczna i dobroć:
\[
\omega_0=\frac{1}{R\sqrt{C_1C_2}}=1607
\]
\[
f_0=\frac{\omega_0}{2\pi}=256Hz
\]
\[
Q=0,5\sqrt{\frac{C_2}{C_1}}=0,707
\]

\newpage

Przeprowadzilismy pomiary w celu wyznaczenia charakterystyki amplitudowo fazowej filtru dolnoprzepustowego:

\begin{table}[!ht]
    \centering
    \begin{tabular}{|l|l|l|l|}
    \hline
        Lp & f[Hz] & k[dB] & faza[$ \degree $] \\ \hline
        1 & 50 & 0 & -16 \\ \hline
        2 & 100 & -0,1 & -30 \\ \hline
        3 & 150 & -0,5 & -50 \\ \hline
        4 & 200 & -1,3 & -70 \\ \hline
        5 & 258 & -3 & -90 \\ \hline
        6 & 300 & -4,6 & -100 \\ \hline
        7 & 350 & -6,5 & -110 \\ \hline
        8 & 500 & -12 & -130 \\ \hline
        9 & 1k & -23,8 & -160 \\ \hline
        10 & 2k & -35 & -170 \\ \hline
        11 & 5k & -50 & -180\\ \hline
    \end{tabular}
    \caption{Dane uzyskane przy użyciu filter.vi}
\end{table}

Widać że, występuje drobna rozbieżność pomiedzy obliczona teoretycznie częstotliwościa graniczna a zmierzoną eksperymentalnie, rozbieżność najprawodpodobnie wynika z tolerancji rezystorów

\begin{figure}[H]
    \centering
    \includegraphics[width=0.7\linewidth]{img/apmdolno.png}
    \caption{Charakterystyka amplitudowa wyznaczona na podstawie danych pomiarowych}
\end{figure}

\begin{figure}[H]
    \centering
    \includegraphics[width=0.7\linewidth]{img/phasedolno.png}
    \caption{Charakterystyka fazowa wyznaczona na podstawie danych pomiarowych}
\end{figure}

\begin{figure}[H]
    \centering
    \includegraphics[width=0.7\linewidth]{img/bodedolnoteoria.png}
    \caption{Charakterystyka fazowa wyznaczona na podstawie danych pomiarowych}
\end{figure}


\section{Filtr górnoprzepustowy}

\begin{figure}[H]
    \centering
    \includegraphics[width=0.5\linewidth]{img/gornoschemat.png}
    \caption{aktywny górnoprzepustowy filtr RC}
\end{figure}

\[
C_1=C_2=C=22nF \qquad R_1=20k\Omega \qquad R_2=10k\Omega
\]

Znajac wartosci elementów z jakich składa sie filtr możemy obliczyć jego częstotliwość graniczna i dobroć:
\[
\omega_0=\frac{1}{C\sqrt{R_1R_2}}=3214
\]
\[
f_0=\frac{\omega_0}{2\pi}=511Hz
\]
\[
Q=0,5\sqrt{\frac{R_1}{R_2}}=0,707
\]

\begin{table}[!ht]
    \centering
    \begin{tabular}{|l|l|l|l|}
    \hline
        Lp & f[Hz] & k[dB] & faza[st] \\ \hline
        1 & 100 & -28,8 & 170 \\ \hline
        2 & 200 & -16,9 & 150 \\ \hline
        3 & 350 & -7,9 & 120 \\ \hline
        4 & 450 & -4,6 & 110 \\ \hline
        5 & 500 & -3,5 & 94 \\ \hline
        6 & 512 & -3,4 & 92 \\ \hline
        7 & 530 & -3 & 89 \\ \hline
        8 & 600 & -2,1 & 79 \\ \hline
        9 & 1000 & -0,4 & 46 \\ \hline
        10 & 3000 & -0,1 & 14 \\ \hline
        11 & 4500 & -0,1 & 9 \\ \hline
        12 & 7000 & -0,1 & 6 \\ \hline
    \end{tabular}
    \caption{Dane uzyskane przy użyciu filter.vi}
\end{table}

\begin{figure}[H]
    \centering
    \includegraphics[width=0.7\linewidth]{img/gornoampl.png}
    \caption{Charakterystyka amplitudowa wyznaczona na podstawie danych pomiarowych}
\end{figure}

\begin{figure}[H]
    \centering
    \includegraphics[width=0.7\linewidth]{img/gornofazowa.png}
    \caption{Charakterystyka fazowa wyznaczona na podstawie danych pomiarowych}
\end{figure}

\begin{figure}[H]
    \centering
    \includegraphics[width=0.7\linewidth]{img/bodegorno.png}
    \caption{Charakterystyka fazowa wyznaczona na podstawie danych pomiarowych}
\end{figure}

\section{Filtr pasmowoprzepustowy}

\begin{figure}[H]
    \centering
    \includegraphics[width=0.5\linewidth]{img/pasmoschemat.png}
    \caption{aktywny pasmowoprzepustowy filtr RC}
\end{figure}

\begin{figure}[H]
    \centering
    \includegraphics[width=0.5\linewidth]{img/scheamtpasmoworzecz.png}
    \caption{rzeczywisty schemat aktywny pasmowoprzepustowy filtr RC}
\end{figure}

\[
C_1=C_2=C=22nF \qquad R_1=20k\Omega \qquad R_3=100k\Omega 
\]

Porównujac schemat rzeczuwistych połączeń widzimy że naszym rezystorem $R_2$ są połączone $R_2$,$R_4$ i potencjometr
\[
R_{2R}=\frac{(R_4+P)R_2}{R_4+P+R_2}
\]
\[
R_2=1k\Omega \qquad R_4=2k\Omega
\]
\[
R_{2\; 1k\Omega}=500\Omega \qquad R_{2\; 5k\Omega}= 875\Omega \qquad R_{2\; 10k\Omega}=923\Omega
\]

\[
\omega_0=\frac{1}{C\sqrt{R_3\frac{R_1R_{2R}}{R_1+R_{2R}}}}=4676
\]
\[
f_0=\frac{\omega_0}{2\pi}=741Hz
\]
\[
\Delta \omega=\frac{2}{CR_3}=909 \qquad \Delta f=\frac{\Delta \omega}{2\pi}=145Hz
\]
\[
f_d=f_0-\Delta f =595Hz \qquad f_g=f_0+\Delta f=886Hz
\]
\[
Q=\frac{\omega_0}{\Delta \omega}=5,123
\]

\begin{figure}[H]
    \centering
    \includegraphics[width=0.6\linewidth]{img/pasmowocharak.png}
\end{figure}


\section{Wnioski}


\end{document}