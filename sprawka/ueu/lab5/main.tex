\documentclass[11pt]{article}

\pdfobjcompresslevel=3    % compress PDF objects
\pdfcompresslevel=9       % compress streams more (0..9)
\pdfminorversion=4


\usepackage[utf8]{inputenc} % remove if using XeLaTeX or LuaLaTeX
\usepackage[T1]{fontenc}

% Layout & graphics
\usepackage[a4paper, total={7.5in, 9in}]{geometry}
\usepackage{graphicx}              % images
\usepackage[dvipsnames,table]{xcolor} % single xcolor invocation (keeps table option)

% Math & symbols
\usepackage{amsmath}
\usepackage{amssymb}
\usepackage{gensymb}

% Typography & layout helpers
\usepackage{microtype}
\usepackage{float}                 % [H] placement
\usepackage{caption}
\usepackage{subcaption}            % modern subfigure support (do NOT load subfig)
\usepackage{multirow}
\usepackage{titlesec}

\usepackage{lmodern}
\usepackage{microtype}

\definecolor{xppblue}{RGB}{37, 150, 190}

\begin{document}

\begin{table}[h]
\centering
\scalebox{1.5}{
\begin{tabular}{|ccll|cl|}
\hline
\multicolumn{4}{|c|}{SPRAWOZDANIE Z LABORATORIUM}                                                                                                                            & \multicolumn{2}{c|}{\multirow{2}{*}{\begin{tabular}[c]{@{}c@{}}rok akademicki:\\ 2025/26\end{tabular}}} \\ \cline{1-4}
\multicolumn{4}{|c|}{\textbf{Układy elektroniki użytkowej}}                                                                                                                          & \multicolumn{2}{c|}{}                                                                                   \\ \hline
\multicolumn{4}{|c|}{\multirow{2}{*}{\textit{Filtry Aktywne}}}                                                                                           & \multicolumn{2}{c|}{\multirow{2}{*}{czwartek 8:00}}                                                \\
\multicolumn{4}{|c|}{}                                                                                                                                                       & \multicolumn{2}{c|}{}                                                                                   \\ \hline
\multicolumn{1}{|c|}{WARiE, AiR, sem 5}          & \multicolumn{3}{c|}{\multirow{2}{*}{\begin{tabular}[c]{@{}c@{}} \underline{1. Jan Andrzejewski}\\2. Mateusz Banaszak\\\end{tabular}}} & \multicolumn{2}{l|}{\multirow{2}{*}{Punkty:}}                                                           \\ \cline{1-1}
\multicolumn{1}{|c|}{30.10.2025}                   & \multicolumn{3}{c|}{}                                                                                                   & \multicolumn{2}{l|}{}                                                                                   \\ \hline
\end{tabular}
}
\end{table}
\vspace{2\baselineskip}

\section{Cel i zakres ćwiczenia}
Celem ćwiczenia jest zapoznanie się z działaniem prostych filtrów aktywnych zrealizowanych
z wykorzystaniem elementów RC i wzmacniacza operacyjnego, a także ze specyfiką pomiarów
filtrów. W ćwiczeniu wykorzystano wirtualne przyrządy pomiarowe opracowane w środowisku
LabVIEW oraz urządzenie ELVIS II.

\section{Filtr dolnoprzepustowy}

\begin{figure}[H]
    \centering
    \includegraphics[width=0.5\linewidth]{img/dolnoschemat.png}
    \caption{aktywny dolnoprzepustowy filtr RC}
\end{figure}

\[
R_1=R_2=R=20k\Omega \qquad C_1=22nF \qquad C_2=44nF
\]

Znajac wartosci elementów z jakich składa sie filtr możemy obliczyć jego częstotliwość graniczna i dobroć:
\[
\omega_0=\frac{1}{R\sqrt{C_1C_2}}=1607
\]
\[
f_0=\frac{\omega_0}{2\pi}=256Hz
\]
\[
Q=0,5\sqrt{\frac{C_2}{C_1}}=0,707
\]

Filtr dolnoprzepustowy przepuszcza sygnały o częstotliwościach niższych od częstotliwości granicznej $f_0$, jednocześnie tłumiąc składowe o częstotliwościach wyższych. W badanym układzie wykorzystano strukturę Sallen-Key, która zapewnia dobrą stabilność przy stosunkowo prostej budowie. Kondensatory $C_1$ i $C_2$ wraz z rezystorami $R_1$ i $R_2$ tworzą sieć kształtującą charakterystykę częstotliwościową, podczas gdy wzmacniacz operacyjny pracuje w konfiguracji wtórnika napięciowego, zapewniając izolację wyjścia od obciążenia.

Przeprowadzilismy pomiary w celu wyznaczenia charakterystyki amplitudowo fazowej filtru dolnoprzepustowego.

\begin{table}[!ht]
    \centering
    \begin{tabular}{|l|l|l|l|}
    \hline
        Lp & f[Hz] & k[dB] & faza[$ \degree $] \\ \hline
        1 & 50 & 0 & -16 \\ \hline
        2 & 100 & -0,1 & -30 \\ \hline
        3 & 150 & -0,5 & -50 \\ \hline
        4 & 200 & -1,3 & -70 \\ \hline
        5 & 258 & -3 & -90 \\ \hline
        6 & 300 & -4,6 & -100 \\ \hline
        7 & 350 & -6,5 & -110 \\ \hline
        8 & 500 & -12 & -130 \\ \hline
        9 & 1k & -23,8 & -160 \\ \hline
        10 & 2k & -35 & -170 \\ \hline
        11 & 5k & -50 & -180\\ \hline
    \end{tabular}
    \caption{Dane uzyskane przy użyciu filter.vi}
\end{table}

Widać że, występuje drobna rozbieżność pomiedzy obliczona teoretycznie częstotliwościa graniczna a zmierzoną eksperymentalnie, różnica wynośi około 2Hz, rozbieżność najprawodpodobniej wynika z tolerancji rezystorów, kondensatorów i niedoskonałości samego wzmacniacza.

\subsection{Analiza wyników dla filtru dolnoprzepustowego}
Analiza danych pomiarowych wykazuje charakterystyczne cechy filtru Butterwortha drugiego rzędu. W paśmie przepustowym (do około 150 Hz) tłumienie nie przekracza 0,5 dB, co potwierdza maksymalną płaskość charakterystyki. Stromość tłumienia po przekroczeniu częstotliwości granicznej wynosi w przybliżeniu 40 dB/dekadę, co odpowiada teoretycznej wartości dla filtru drugiego rzędu (20 dB/dekadę na każdy biegun). 

Przesunięcie fazowe wynosi -90° dla częstotliwości granicznej, co jest zgodne z teorią dla filtru Butterwortha drugiego rzędu. Przy niskich częstotliwościach faza dąży do 0°, natomiast dla częstotliwości znacznie przekraczających $f_0$ stabilizuje się na poziomie -180°. Taka charakterystyka fazowa jest typowa dla układów drugiego rzędu i wskazuje na dwukrotne całkowanie sygnału w zakresie wysokich częstotliwości.

\begin{figure}[H]
    \centering
    \includegraphics[width=0.7\linewidth]{img/apmdolno.png}
    \caption{Charakterystyka amplitudowa wyznaczona na podstawie danych pomiarowych}
\end{figure}

\begin{figure}[H]
    \centering
    \includegraphics[width=0.7\linewidth]{img/phasedolno.png}
    \caption{Charakterystyka fazowa wyznaczona na podstawie danych pomiarowych}
\end{figure}

\begin{figure}[H]
    \centering
    \includegraphics[width=0.7\linewidth]{img/dolnobode.png}
    \caption{Charakterystyka fazowa wyznaczona na podstawie danych pomiarowych}
\end{figure}


\section{Filtr górnoprzepustowy}

\begin{figure}[H]
    \centering
    \includegraphics[width=0.5\linewidth]{img/gornoschemat.png}
    \caption{aktywny górnoprzepustowy filtr RC}
\end{figure}

\[
C_1=C_2=C=22nF \qquad R_1=20k\Omega \qquad R_2=10k\Omega
\]

Znajac wartosci elementów z jakich składa sie filtr możemy obliczyć jego częstotliwość graniczna i dobroć:
\[
\omega_0=\frac{1}{C\sqrt{R_1R_2}}=3214
\]
\[
f_0=\frac{\omega_0}{2\pi}=511Hz
\]
\[
Q=0,5\sqrt{\frac{R_1}{R_2}}=0,707
\]

Filtr górnoprzepustowy stanowi funkcjonalne odwrócenie filtru dolnoprzepustowego, przepuszcza sygnały o częstotliwościach wyższych od częstotliwości granicznej, tłumiąc składowe o niskich czestotliwościach. W badanym układzie kondensatory i rezystory zamieniają swoje role w porównaniu z filtrem dolnoprzepustowym: kondensatory znajdują się w torze sygnałowym, natomiast rezystory w pętlach sprzężenia zwrotnego. Taka konfiguracja zapewnia rosnące impedancje dla malejących częstotliwości, co skutkuje tłumieniem składowych o nieskiej czestotliwości.

\begin{table}[!ht]
    \centering
    \begin{tabular}{|l|l|l|l|}
    \hline
        Lp & f[Hz] & k[dB] & faza[st] \\ \hline
        1 & 100 & -28,8 & 170 \\ \hline
        2 & 200 & -16,9 & 150 \\ \hline
        3 & 350 & -7,9 & 120 \\ \hline
        4 & 450 & -4,6 & 110 \\ \hline
        5 & 500 & -3,5 & 94 \\ \hline
        6 & 512 & -3,4 & 92 \\ \hline
        7 & 530 & -3 & 89 \\ \hline
        8 & 600 & -2,1 & 79 \\ \hline
        9 & 1000 & -0,4 & 46 \\ \hline
        10 & 3000 & -0,1 & 14 \\ \hline
        11 & 4500 & -0,1 & 9 \\ \hline
        12 & 7000 & -0,1 & 6 \\ \hline
    \end{tabular}
    \caption{Dane uzyskane przy użyciu filter.vi}
\end{table}

\subsection{Analiza wyników dla filtru górnoprzepustowego}
Charakterystyka amplitudowa filtru górnoprzepustowego wykazuje oczekiwaną stromość -40 dB/dekadę w zakresie częstotliwości poniżej $f_0$. Dla częstotliwości 100 Hz, będącej około jedną piątą częstotliwości granicznej, tłumienie wynosi -28,8 dB, co dobrze koreluje z teoretycznymi przewidywaniami. W paśmie przepustowym (powyżej 1 kHz) charakterystyka stabilizuje się na poziomie bliskim 0 dB z tłumieniem nieprzekraczającym -0,4 dB.

Charakterystyka fazowa filtru górnoprzepustowego jest lustrzanym odbiciem charakterystyki filtru dolnoprzepustowego jak chodzi o kształt. Dla niskich częstotliwości przesunięcie fazowe zbliża się do +180°, przy częstotliwości granicznej wynosi około +90°, a dla wysokich częstotliwości dąży asymptotycznie do 0°. Zmierzona wartość fazy około 92° dla częstotliwości 512 Hz potwierdza prawidłowość doboru elementów i zgodność z teorią.

\begin{figure}[H]
    \centering
    \includegraphics[width=0.7\linewidth]{img/gornoampl.png}
    \caption{Charakterystyka amplitudowa wyznaczona na podstawie danych pomiarowych}
\end{figure}

\begin{figure}[H]
    \centering
    \includegraphics[width=0.7\linewidth]{img/gornofazowa.png}
    \caption{Charakterystyka fazowa wyznaczona na podstawie danych pomiarowych}
\end{figure}

\begin{figure}[H]
    \centering
    \includegraphics[width=0.7\linewidth]{img/bodegorno.png}
    \caption{Charakterystyka fazowa wyznaczona na podstawie danych pomiarowych}
\end{figure}

\section{Filtr pasmowoprzepustowy}

\begin{figure}[H]
    \centering
    \includegraphics[width=0.5\linewidth]{img/pasmoschemat.png}
    \caption{aktywny pasmowoprzepustowy filtr RC}
\end{figure}


\[
C_1=C_2=C=22nF \qquad R_1=20k\Omega \qquad R_3=100k\Omega \qquad R_2=1k\Omega \qquad R_4=2k\Omega
\]

\[
\omega_{0}=\frac{1}{C\sqrt{R_3\frac{R_1R_{2}}{R_1+R_{2}}}}=4653
\]
\[
f_0=\frac{\omega_0}{2\pi}=741Hz
\]
\[
\Delta \omega=\frac{2}{CR_3}=909 \qquad \Delta f=\frac{\Delta \omega}{2\pi}=145Hz
\]
\[
f_d=f_0-\Delta f =595Hz \qquad f_g=f_0+\Delta f=886Hz
\]
\[
Q=\frac{\omega_0}{\Delta \omega}=5,123
\]

\subsection{Charakterystyka filtru pasmowoprzepustowego}
Filtr pasmowoprzepustowy łączy właściwości filtrów dolno- i górnoprzepustowego, przepuszczając jedynie sygnały w określonym zakresie częstotliwości. Badany układ wykorzystuje dodatnie i ujemne sprzężenie zwrotne do uzyskania charakterystyki rezonansowej. Potencjometr P umożliwia regulację szerokości pasma przepustowego przez zmianę wartości zastępczej rezystora $R_2$, co bezpośrednio wpływa na dobroć Q filtru.

Wysoka wartość dobroci Q = 5,123 wskazuje na wąskie pasmo przepustowe i selektywną charakterystykę. Częstotliwości graniczne $f_d = 595$ Hz i $f_g = 886$ Hz definiują zakres, w którym tłumienie nie przekracza -3 dB.

\begin{figure}[H]
    \centering
    \includegraphics[width=0.6\linewidth]{img/image.png}
\end{figure}

\section{Wpływ dobroci na przebieg charakterystyki}
Dobroć Q jest kluczowym parametrem określającym kształt charakterystyki filtru. Dla filtrów drugiego rzędu wartości Q można podzielić na trzy zakresy:

\textbf{Q < 0,5:} Układ jest przetłumiony (overdamped), charakterystyka jest bardzo łagodna, brak wyraźnej częstotliwości granicznej. Tłumienie narasta bardzo powoli, co skutkuje słabą selektywnością.

\textbf{Q = 0,707:} Charakterystyka Butterwortha - optymalny kompromis. Maksymalna płaskość w paśmie przepustowym przy zachowaniu stromości -40 dB/dekadę. Brak przesterowania i rezonansu.

\textbf{Q > 0,707:} Pojawia się pik rezonansowy przed częstotliwością graniczną. Im wyższe Q, tym wyższy pik i węższa charakterystyka. Dla Q > 5 układ może stać się niestabilny i generować oscylacje.

W praktycznych zastosowaniach wybór wartości Q zależy od wymagań: w filtrach audio preferuje się Q = 0,707 dla minimalizacji zniekształceń, podczas gdy w selektywnych filtrach pomiarowych stosuje się wysokie wartości Q dla lepszej separacji sygnałów.

\begin{figure}[H]
    \centering
    \includegraphics[width=0.5\linewidth]{img/dobroc.png}
    \caption{Stromo charakterystyk dla różnych dobroci}
\end{figure}


\subsection{Parametry filtru dla róznych charakterystyk}

Przeprowadzimy modyfikacje charakterystyki filtru dolnoprzepustowego, dobór kondensatorów przeprowadzimy zgodnie z wzorami:
\[
C_1=\frac{a_1}{2\pi f_0 R}
\]
\[
C_1=\frac{b_1}{2\pi f_0 R}
\]
gdzie $a_1$ i $b_1$ są stałymi z tabel.

\subsection*{Parametry poczatkowe}
\[
R_1=R_2=20k\Omega \qquad f_0=256Hz 
\]

\subsection*{Charakterystyka Bessela}
\[
C_1=\frac{1,3617}{2\pi \cdot 256 \cdot 20000}=42,3nF
\]
\[
C_2=\frac{0,6180}{2\pi \cdot 256 \cdot 20000}=19,2nF
\]

Filtry Bessela charakteryzują się liniowym przesunięciem fazowym w paśmie przepustowym. Są szczególnie przydatne w aplikacjach, gdzie zachowanie kształtu impulsu jest krytyczne, takich jak transmisja danych czy przetwarzanie sygnałów impulsowych.

\subsection*{Charakterystyka Czebyszewa}
\[
C_1=\frac{1,4256}{2\pi \cdot 256 \cdot 20000}=44,3nF
\]
\[
C_1=\frac{0,6265}{2\pi \cdot 256 \cdot 20000}=19,4nF
\]

Filtry Czebyszewa oferują bardziej stromą charakterystykę niż filtry Butterwortha, kosztem wprowadzenia falowania w paśmie przepustowym. Typ Czebyszewa I ma falowanie w paśmie przepustowym, podczas gdy typ II ma falowanie w paśmie zaporowym. Takie filtry są stosowane tam, gdzie wymagana jest ostra separacja częstotliwości przy akceptowalnym poziomie zniekształceń.

\section{Wnioski}

Przebadane aktywne filtry  dolnoprzepustowy, górnoprzepustowy oraz pasmowoprzepustowy działają zgodnie z założeniami teoretycznymi. Uzyskane charakterystyki pomiarowe potwierdzają cechy filtrów Butterwortha, które wyróżniają się maksymalnie płaskim pasmem przepustowym. Niewielkie różnice między wynikami teoretycznymi a pomiarowymi wynikają z czynników praktycznych, takich jak tolerancje elementów, niedoskonałości wzmacniaczy operacyjnych oraz ograniczenia aparatury pomiarowej.

Wykorzystanie struktury Sallen-Key we wszystkich badanych układach zapewniło dobrą stabilność i prostotę implementacji. Wzmacniacze operacyjne zastosowane w układach wykazały zadowalające parametry w zakresie badanych częstotliwości, nie wprowadzając znaczących zniekształceń nieliniowych ani ograniczeń pasma.


\end{document}