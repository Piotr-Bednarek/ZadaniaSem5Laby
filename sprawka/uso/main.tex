\documentclass[11pt]{article}

\pdfobjcompresslevel=3    % compress PDF objects
\pdfcompresslevel=9       % compress streams more (0..9)
\pdfminorversion=4


\usepackage[utf8]{inputenc} % remove if using XeLaTeX or LuaLaTeX
\usepackage[T1]{fontenc}

% Layout & graphics
\usepackage[a4paper, total={7.5in, 9in}]{geometry}
\usepackage{graphicx}
\usepackage{wrapfig}              % images
\usepackage[dvipsnames,table]{xcolor} % single xcolor invocation (keeps table option)

% Math & symbols
\usepackage{amsmath}
\usepackage{amssymb}
\usepackage{gensymb}

% Typography & layout helpers
\usepackage{microtype}
\usepackage{float}                 % [H] placement
\usepackage{caption}
\usepackage{subcaption}            % modern subfigure support (do NOT load subfig)
\usepackage{multirow}
\usepackage{titlesec}

\usepackage{lmodern}
\usepackage{microtype}

\definecolor{xppblue}{RGB}{37, 150, 190}

\title{Sprawozdanie z laboratorium \par Układy sterowania optymalnego}
\author{Jan Andrzejewski 159512}
\date{}

\begin{document}

\maketitle

\newpage

\section{Lab 3-4}
\subsection{Układ RC}

Pod rozważania brany jest układ RC przedstawiony an Figure 1. Przyjęto że:


\begin{figure}[H]
    \centering
    \includegraphics[width=0.5\linewidth]{img/ukladrc.png}
    \caption{Układ RC}
\end{figure}
\[
R_1=1\Omega \; C_1=1F
\]
\[
 R_2=1\Omega \;C_2=2F
\]
\[
R_3=1\Omega \: C_3=3F
\]


równania stanu:
\[
\dot{x}=
\begin{bmatrix}
 -\dfrac{1}{R_1 C_1} & 0 & 0 \\
 0 & -\dfrac{1}{R_2 C_2} & 0 \\
 0 & 0 & -\dfrac{1}{R_3 C_3}
\end{bmatrix}x + 
\begin{bmatrix}
    -\dfrac{1}{R_1 C_1} \\
    -\dfrac{1}{R_2 C_2}\\
    -\dfrac{1}{R_3 C_3} \\
\end{bmatrix}u
\]

\subsection{Teoretyczne badanie sterowalności}
\[
K=\begin{bmatrix}
    A & AB & A^2 B\\
\end{bmatrix}=\begin{bmatrix}
    1 & -1 & 1 \\
 0.5 & -0.25 & 0.125 \\
 \frac{1}{3} & -0.(1) & 0.(037)
\end{bmatrix}
\]

\[
rank(K)=3 
\]
\[
n=3
\]

System jest sterowalny


\subsection{Doswiadczalne badanie sterowalności}

\begin{figure}[H]
    \centering
    \includegraphics[width=0.5\linewidth]{img/wykres.png}
    \caption{Układ RC}
\end{figure}

\subsection{Postać sterowalna}

sys2 

\subsection{}




\end{document}