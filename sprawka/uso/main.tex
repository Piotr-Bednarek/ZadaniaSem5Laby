\documentclass[11pt]{article}

\pdfobjcompresslevel=3    % compress PDF objects
\pdfcompresslevel=9       % compress streams more (0..9)
\pdfminorversion=4


\usepackage[utf8]{inputenc} % remove if using XeLaTeX or LuaLaTeX
\usepackage[T1]{fontenc}

% Layout & graphics
\usepackage[a4paper, total={7.5in, 9in}]{geometry}
\usepackage{graphicx}
\usepackage{wrapfig}              % images
\usepackage[dvipsnames,table]{xcolor} % single xcolor invocation (keeps table option)

% Math & symbols
\usepackage{amsmath}
\usepackage{amssymb}
\usepackage{gensymb}

% Typography & layout helpers
\usepackage{microtype}
\usepackage{float}                 % [H] placement
\usepackage{caption}
\usepackage{subcaption}            % modern subfigure support (do NOT load subfig)
\usepackage{multirow}
\usepackage{titlesec}

\usepackage{lmodern}
\usepackage{microtype}
\usepackage{listings}

\definecolor{xppblue}{RGB}{37, 150, 190}

\title{Sprawozdanie z laboratorium \par Układy sterowania optymalnego}
\author{Jan Andrzejewski 159512}
\date{}

\begin{document}

\maketitle

\newpage

\section{Lab 3-4}
\subsection{Układ RC}

Pod rozważania brany jest układ RC przedstawiony an Figure 1. Przyjęto że:


\begin{figure}[H]
    \centering
    \includegraphics[width=0.5\linewidth]{img/ukladrc.png}
    \caption{Układ RC}
\end{figure}
\[
R_1=1\Omega \; C_1=1F
\]
\[
 R_2=1\Omega \;C_2=2F
\]
\[
R_3=1\Omega \: C_3=3F
\]


równania stanu:
\[
\dot{x}=
\begin{bmatrix}
 -\dfrac{1}{R_1 C_1} & 0 & 0 \\
 0 & -\dfrac{1}{R_2 C_2} & 0 \\
 0 & 0 & -\dfrac{1}{R_3 C_3}
\end{bmatrix}x + 
\begin{bmatrix}
    -\dfrac{1}{R_1 C_1} \\
    -\dfrac{1}{R_2 C_2}\\
    -\dfrac{1}{R_3 C_3} \\
\end{bmatrix}u
\]

\subsection{Teoretyczne badanie sterowalności}
\[
K=\begin{bmatrix}
    A & AB & A^2 B\\
\end{bmatrix}=\begin{bmatrix}
    1 & -1 & 1 \\
 0.5 & -0.25 & 0.125 \\
 \frac{1}{3} & -0.(1) & 0.(037)
\end{bmatrix}
\]

\[
rank(K)=3 
\]
\[
n=3
\]

System jest sterowalny

\subsection{Postać sterowalna}

Aby doprowadzic system do postaci sterowalnej skorzystamy z zaleznosci:
\[
\phi(s)=det(Is-A)=s^n+\sum_{i=0}^{n-1}a_is^i
\]

w przypadku układu RC:
\[
\phi(s)=det \bigg(\begin{bmatrix}
    s & 0 & 0 \\
 0 & s & 0 \\
 0 & 0 & s
\end{bmatrix}-\begin{bmatrix}
    -1 & 0 & 0 \\
 0 & -\frac{1}{2} & 0 \\
 0 & 0 & -\frac{1}{3}
\end{bmatrix} \bigg)=(s+1)(s+\frac{1}{2})(s+\frac{1}{3})=s^2+\frac{11}{6}s^2+s+\frac{1}{6}
\]

Postać sterowalna układu RC wyglada nastepujaco:

\[
\dot{x}=\begin{bmatrix}
    0 & 1 & 0 \\
 0 & 0 & 1 \\
 \frac{1}{6} & 1 & \frac{11}{6}
\end{bmatrix}x+\begin{bmatrix}
    0\\
    0\\
    1
\end{bmatrix}u
\]


\subsection{Porównanie obiektu opisanego postacią normalna regulatorowa i sterowalną}

\begin{figure}[H]
\centering
\begin{subfigure}{.5\textwidth}
  \centering
  \includegraphics[width=1\linewidth]{img/jednostkowy.png}
  \caption{kanoniczna}
  \label{fig:sub1}
\end{subfigure}%
\begin{subfigure}{.5\textwidth}
  \centering
  \includegraphics[width=1\linewidth]{img/jednostkowySTER.png}
  \caption{sterowalna}
  \label{fig:sub2}
\end{subfigure}
\caption{Wykresy przebiegów zmiennych stanu dla pobudzenia skokiem jednostkowym}
\label{fig:test}
\end{figure}

\begin{figure}[H]
\centering
\begin{subfigure}{.5\textwidth}
  \centering
  \includegraphics[width=1\linewidth]{img/jednostkowy2.png}
  \caption{A subfigure}
  \label{fig:sub1}
\end{subfigure}%
\begin{subfigure}{.5\textwidth}
  \centering
  \includegraphics[width=1\linewidth]{img/jednostkowy2Ster.png}
  \caption{A subfigure}
  \label{fig:sub2}
\end{subfigure}
\caption{Wykresy przebiegów zmiennych stanu dla pobudzenia skokiem jednostkowym o amplitudzie 2}
\label{fig:test}
\end{figure}

\begin{figure}[H]
\centering
\begin{subfigure}{.5\textwidth}
  \centering
  \includegraphics[width=1\linewidth]{img/sin.png}
  \caption{A subfigure}
  \label{fig:sub1}
\end{subfigure}%
\begin{subfigure}{.5\textwidth}
  \centering
  \includegraphics[width=1\linewidth]{img/sinSter.png}
  \caption{A subfigure}
  \label{fig:sub2}
\end{subfigure}
\caption{Wykresy przebiegów zmiennych stanu dla pobudzenia sinusem}
\label{fig:test}
\end{figure}

Obie reprezentacje opisuja system w sposob jednakowy sa przeciez tylko kombinacjami liniowymi 

\subsection{Lokowanie biegunów}



\begin{lstlisting}

    A=oryginalne A ukladu RC
    B=oryginalne B ukladu RC

    desired=[-1,-2,-5] #oczekiwane bieguny

    res = place_poles(A,B,desired)
    K=res.gain_matrix

    print("Using scipy.signal.place_poles:")
    print(f"K = {K}")
    print(f"Achieved poles: {res.computed_poles}")
\end{lstlisting}

\[
K=\begin{bmatrix}
    1 & -1 & 1 \\
 0.5 & -0.25 & 0.125 \\
 0.333 & -0.111 & 0.037
\end{bmatrix}
\]

macierz wzmocnien spelniajaca dane bieguny

\[
K = [[ 5.17069381e-14 -8.10000000e+01  1.40000000e+02]]
\]
nie wiem jeszcze co to jest 

\end{document}