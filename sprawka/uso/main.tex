\documentclass[11pt]{article}

\pdfobjcompresslevel=3    % compress PDF objects
\pdfcompresslevel=9       % compress streams more (0..9)
\pdfminorversion=4


\usepackage[utf8]{inputenc} % remove if using XeLaTeX or LuaLaTeX
\usepackage[T1]{fontenc}
\usepackage{babel}[Polish]

% Layout & graphics
\usepackage[a4paper, total={7.5in, 9in}]{geometry}
\usepackage{graphicx}
\usepackage{wrapfig}              % images
\usepackage[dvipsnames,table]{xcolor} % single xcolor invocation (keeps table option)

% Math & symbols
\usepackage{amsmath}
\usepackage{amssymb}
\usepackage{gensymb}

% Typography & layout helpers
\usepackage{microtype}
\usepackage{float}                 % [H] placement
\usepackage{caption}
\usepackage{subcaption}            % modern subfigure support (do NOT load subfig)
\usepackage{multirow}
\usepackage{titlesec}

\usepackage{lmodern}
\usepackage{microtype}
\usepackage{listings}

\definecolor{xppblue}{RGB}{37, 150, 190}

\title{Sprawozdanie 1 z laboratorium \par Układy sterowania optymalnego}
\author{Jan Andrzejewski 159512}
\date{}

\begin{document}

\maketitle

\newpage

\section{Lab 3-4}
\subsection{Układ RC}

Pod rozważania brany jest układ RC przedstawiony an Figure 1. Przyjęto że:


\begin{figure}[H]
    \centering
    \includegraphics[width=0.5\linewidth]{img/ukladrc.png}
    \caption{Układ RC}
\end{figure}
\[
R_1=1\Omega \qquad C_1=1F
\]
\[
 R_2=1\Omega \qquad C_2=2F
\]
\[
R_3=1\Omega \qquad C_3=3F
\]


równania stanu:
\[
\dot{x}=
\begin{bmatrix}
 -\dfrac{1}{R_1 C_1} & 0 & 0 \\
 0 & -\dfrac{1}{R_2 C_2} & 0 \\
 0 & 0 & -\dfrac{1}{R_3 C_3}
\end{bmatrix}x + 
\begin{bmatrix}
    -\dfrac{1}{R_1 C_1} \\
    -\dfrac{1}{R_2 C_2}\\
    -\dfrac{1}{R_3 C_3} \\
\end{bmatrix}u
\]

\subsection{Teoretyczne badanie sterowalności}
\[
K=\begin{bmatrix}
    A & AB & A^2 B\\
\end{bmatrix}=\begin{bmatrix}
    1 & -1 & 1 \\
 0.5 & -0.25 & 0.125 \\
 0.(3) & -0.(1) & 0.(037)
\end{bmatrix}
\]

\[
rank(K)=3 
\]
\[
n=3
\]

Rząd macierzy Kalmana jest równy 3, zgadza sie z rozmiarem macierzy A co swiadczy o tym że system jest sterowalny.

\subsection{Postać sterowalna}

Aby doprowadzic system do postaci sterowalnej skorzystamy z zaleznosci:
\[
\phi(s)=det(Is-A)=s^n+\sum_{i=0}^{n-1}a_is^i
\]

w przypadku układu RC:
\[
\phi(s)=det \bigg(\begin{bmatrix}
    s & 0 & 0 \\
 0 & s & 0 \\
 0 & 0 & s
\end{bmatrix}-\begin{bmatrix}
    -1 & 0 & 0 \\
 0 & -\frac{1}{2} & 0 \\
 0 & 0 & -\frac{1}{3}
\end{bmatrix} \bigg)=(s+1)(s+\frac{1}{2})(s+\frac{1}{3})=s^2+\frac{11}{6}s^2+s+\frac{1}{6}
\]

Postać sterowalna układu RC wyglada nastepujaco:

\[
\dot{x}=\begin{bmatrix}
    0 & 1 & 0 \\
 0 & 0 & 1 \\
 -\frac{1}{6} & -1 & -\frac{11}{6}
\end{bmatrix}x+\begin{bmatrix}
    0\\
    0\\
    1
\end{bmatrix}u
\]


\subsection{Porównanie obiektu opisanego postacią normalna regulatorowa i sterowalną}

\subsubsection{Wyjścia}

Dla danego układu RC zaimplementowano postać sterowalna i zbadano zgodność wyjść z orginalna postacią 

\begin{figure}[H]
\centering
\begin{subfigure}{.5\textwidth}
  \centering
  \includegraphics[width=1\linewidth]{img/wyjsciajednostkowy.png}
  \caption{og}
  \label{fig:sub1}
\end{subfigure}%
\begin{subfigure}{.5\textwidth}
  \centering
  \includegraphics[width=1\linewidth]{img/wyjsciajednostkowysterowalna.png}
  \caption{sterowalna}
  \label{fig:sub2}
\end{subfigure}
\caption{Wykresy przebiegów zmiennych stanu dla pobudzenia skokiem jednostkowym}
\label{fig:test}
\end{figure}

Zgodnie z założeniami zmiana reprezentacji nie zmieniła przebiegu wyjść, dowodzi to temu że, każdy system liniowy można zapisac w nieskończonej ilosci kombinacji zapisu macierzowego równań stanu.

Pomimo że, zmiana zapisu nie wpłyneła na przebiegi wyjść zmieniły sie przebiegi zmiennych stanu, wniosek: zmiana macierzy opisujacych system nie ma wpływu na charakter wyjścia zmienia natomiast sposób w jaki osiągamy wyjscie czyli kombinajce zmiennych stanu składających sie na wyjście 

\begin{figure}[H]
\centering
\begin{subfigure}{.5\textwidth}
  \centering
  \includegraphics[width=1\linewidth]{img/zmiennestanuog.png}
  \caption{og zmienne stanu}
  \label{fig:sub1}
\end{subfigure}%
\begin{subfigure}{.5\textwidth}
  \centering
  \includegraphics[width=1\linewidth]{img/zmiennestanuster.png}
  \caption{zmienne stanu reprezentacji sterowalnej}
  \label{fig:sub2}
\end{subfigure}
\caption{Wykresy przebiegów wyjść dla pobudzenia sinusem}
\label{fig:test}
\end{figure}

\begin{figure}[H]
\centering
\begin{subfigure}{.5\textwidth}
  \centering
  \includegraphics[width=1\linewidth]{img/zmiennestanuogsin.png}
  \caption{og zmienne stanu}
  \label{fig:sub1}
\end{subfigure}%
\begin{subfigure}{.5\textwidth}
  \centering
  \includegraphics[width=1\linewidth]{img/zmiennestanustersin.png}
  \caption{zmienne stanu reprezentacji sterowalnej}
  \label{fig:sub2}
\end{subfigure}
\caption{Wykresy przebiegów wyjść dla pobudzenia sinusem}
\label{fig:test}
\end{figure}

\subsection{Lokowanie biegunów}

Za cel obrano ulokowanie biegunów w lewej półpłaszczyznie co oznacza że układ ma byc stabilny. Nalży spodziewac sie ze układ pozbawiony wejscia i wytracony z równowangi powróci do stanu równowagi 

\[
s_1=-1 \qquad s_2=-2 \qquad s_3=-5
\]

Macierz wzmocnien wyznaczona dla takich biegunów:

\[
K = \begin{bmatrix} 9.83333333 & 16 & 6.16666667 \\ \end{bmatrix}
\]

\begin{figure}[H]
    \centering
    \includegraphics[width=0.7\linewidth]{img/zamkniety.png}
    \caption{Układ RC}
\end{figure}

Układ wytrącono z równowagi w postaci warunków początkowych wynaszacych $\dot{x} = \begin{bmatrix} 1 & 0.5 & -0.5 \\ \end{bmatrix}$







\end{document}