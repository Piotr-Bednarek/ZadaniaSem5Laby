\documentclass[11pt]{article}

\pdfobjcompresslevel=3    % compress PDF objects
\pdfcompresslevel=9       % compress streams more (0..9)
\pdfminorversion=4


\usepackage[utf8]{inputenc} % remove if using XeLaTeX or LuaLaTeX
\usepackage[T1]{fontenc}
\usepackage{polski}[babel]

% Layout & graphics
\usepackage[a4paper, total={7.5in, 9in}]{geometry}
\usepackage{graphicx}
\usepackage{wrapfig}              % images
\usepackage[dvipsnames,table]{xcolor} % single xcolor invocation (keeps table option)

% Math & symbols
\usepackage{amsmath}
\usepackage{amssymb}
\usepackage{gensymb}

% Typography & layout helpers
\usepackage{microtype}
\usepackage{float}                 % [H] placement
\usepackage{caption}
\usepackage{subcaption}            % modern subfigure support (do NOT load subfig)
\usepackage{multirow}
\usepackage{titlesec}

\usepackage{lmodern}
\usepackage{microtype}
\usepackage{listings}
\usepackage{xcolor}

\lstdefinestyle{MyStyle}{
    basicstyle=\ttfamily\small, % Font family and size
    keywordstyle=\color{blue}\bfseries, % Style for keywords
    commentstyle=\color{green!60!black}\itshape, % Style for comments
    %stringstyle=\color{red}, % Style for strings
    numberstyle=\tiny\color{gray}, % Style for line numbers
    backgroundcolor=\color{gray!10}, % Background color
    frame=single, % Add a frame around the code block
    breaklines=true, % Allows long lines to wrap
    language=Java, % Default language
    numbers=left, % Line numbers on the left
    showstringspaces=false % Don't show spaces in strings
}

\lstset{style=MyStyle}

\definecolor{xppblue}{RGB}{37, 150, 190}

\title{Sprawozdanie 1 z laboratorium \par Układy sterowania optymalnego}
\author{Jan Andrzejewski 159512}
\date{}

\begin{document}

\maketitle

\tableofcontents

\newpage

\section{Ćwiczenie 3-4}
\subsection{Układ RC}

Pod rozwagę brany jest układ RC przedstawiony na Rysunku 1.


\begin{figure}[H]
    \centering
    \includegraphics[width=0.5\linewidth]{img/ukladrc.png}
    \caption{Układ RC}
\end{figure}

Gdzie:
\[
R_1=1\Omega \qquad C_1=1F
\]
\[
 R_2=1\Omega \qquad C_2=2F
\]
\[
R_3=1\Omega \qquad C_3=3F
\]

Przyjęto, że: $x1 = u_{c1}$, $x2 = u_{c2}$, $x3 = u_{c3}$.

Równania stanu:
\[
\dot{x}=
\begin{bmatrix}
 -\dfrac{1}{R_1 C_1} & 0 & 0 \\
 0 & -\dfrac{1}{R_2 C_2} & 0 \\
 0 & 0 & -\dfrac{1}{R_3 C_3}
\end{bmatrix}x + 
\begin{bmatrix}
    -\dfrac{1}{R_1 C_1} \\
    -\dfrac{1}{R_2 C_2}\\
    -\dfrac{1}{R_3 C_3} \\
\end{bmatrix}u
\]

\subsection{Teoretyczne badanie sterowalności}
\[
K=\begin{bmatrix}
    A & AB & A^2 B\\
\end{bmatrix}=\begin{bmatrix}
    1 & -1 & 1 \\
 0.5 & -0.25 & 0.125 \\
 0.(3) & -0.(1) & 0.(037)
\end{bmatrix}
\]

\[
rank(K)=3 
\]
\[
n=3
\]

Rząd macierzy Kalmana jest równy 3, zgadza się z rozmiarem macierzy A, co świadczy o tym, że system jest sterowalny.

\subsection{Postać sterowalna}

Aby doprowadzić system do postaci sterowalnej, skorzystamy z zależności:
\[
\phi(s)=det(Is-A)=s^n+\sum_{i=0}^{n-1}a_is^i
\]

W przypadku układu RC:
\[
\phi(s)=det \bigg(\begin{bmatrix}
    s & 0 & 0 \\
 0 & s & 0 \\
 0 & 0 & s
\end{bmatrix}-\begin{bmatrix}
    -1 & 0 & 0 \\
 0 & -\frac{1}{2} & 0 \\
 0 & 0 & -\frac{1}{3}
\end{bmatrix} \bigg)=(s+1)(s+\frac{1}{2})(s+\frac{1}{3})=s^2+\frac{11}{6}s^2+s+\frac{1}{6}
\]

Postać sterowalna układu RC wygląda następująco:

\[
\dot{x}=\begin{bmatrix}
    0 & 1 & 0 \\
 0 & 0 & 1 \\
 -\frac{1}{6} & -1 & -\frac{11}{6}
\end{bmatrix}x+\begin{bmatrix}
    0\\
    0\\
    1
\end{bmatrix}u
\]


\subsection{Porównanie obiektu opisanego postacią normalną regulatorową i sterowalną}


Dla danego układu RC zaimplementowano postać sterowalną i zbadano zgodność wyjść z oryginalną postacią 

\begin{figure}[H]
\centering
\begin{subfigure}{.5\textwidth}
  \centering
  \includegraphics[width=1\linewidth]{img/wyjsciajednostkowy.png}
  \caption{Oryginalna}
  \label{fig:sub1}
\end{subfigure}%
\begin{subfigure}{.5\textwidth}
  \centering
  \includegraphics[width=1\linewidth]{img/wyjsciajednostkowysterowalna.png}
  \caption{Sterowalna}
  \label{fig:sub2}
\end{subfigure}
\caption{Wykresy przebiegów wyjść dla pobudzenia skokiem jednostkowym}
\label{fig:test}
\end{figure}

Zgodnie z założeniami zmiana reprezentacji nie zmieniła przebiegu wyjść. Dowodzi to, że każdy system liniowy można zapisać w nieskończonej ilości kombinacji zapisu macierzowego równań stanu.

Pomimo że zmiana zapisu nie wpłynęła na przebiegi wyjść, zmieniły się przebiegi zmiennych stanu. Wniosek: zmiana macierzy opisujących system nie ma wpływu na charakter wyjścia, zmienia natomiast sposób, w jaki osiągamy wyjście, czyli kombinacje zmiennych stanu składających się na wyjście 

\begin{figure}[H]
\centering
\begin{subfigure}{.5\textwidth}
  \centering
  \includegraphics[width=1\linewidth]{img/zmiennestanuog.png}
  \caption{Oryginalne zmienne stanu}
  \label{fig:sub1}
\end{subfigure}%
\begin{subfigure}{.5\textwidth}
  \centering
  \includegraphics[width=1\linewidth]{img/zmiennestanuster.png}
  \caption{Zmienne stanu reprezentacji sterowalnej}
  \label{fig:sub2}
\end{subfigure}
\caption{Wykresy przebiegów zmiennych stanu dla pobudzenia skokiem o amplitudzie 2}
\label{fig:test}
\end{figure}

\begin{figure}[H]
\centering
\begin{subfigure}{.5\textwidth}
  \centering
  \includegraphics[width=1\linewidth]{img/zmiennestanuogsin.png}
  \caption{Oryginalne zmienne stanu}
  \label{fig:sub1}
\end{subfigure}%
\begin{subfigure}{.5\textwidth}
  \centering
  \includegraphics[width=1\linewidth]{img/zmiennestanustersin.png}
  \caption{Zmienne stanu reprezentacji sterowalnej}
  \label{fig:sub2}
\end{subfigure}
\caption{Wykresy przebiegów zmiennych stanu dla pobudzenia sinusem}
\label{fig:test}
\end{figure}


Przeprowadzone symulacje jasno dowodzą relacji pomiędzy postacią 
normalną regulatorową a postacią sterowalną. Zmiana reprezentacji układu nie 
zmienia jego obserwowalnego zachowania. Zmienia jedynie wewnętrzny przebieg zmiennych stanu, 
które z fizykalnych wielkości zmieniają się w ich liniowe kombinacje.  


\subsection{Lokowanie biegunów}

Za cel obrano ulokowanie biegunów w lewej półpłaszczyźnie, co oznacza, że układ ma być stabilny. Należy spodziewać się, że układ pozbawiony wejścia i wytrącony z równowagi powróci do stanu równowagi 

\[
s_1=-1 \qquad s_2=-2 \qquad s_3=-5
\]

Macierz wzmocnień wyznaczono przy użyciu funkcji place\_poles:
\begin{lstlisting}
  desired=[-1,-2,-5]

  res = scipy.signal.place_poles(An,Bn,desired)
  K=res.gain_matrix
\end{lstlisting}

\[
K = \begin{bmatrix} 9.83333333 & 16 & 6.16666667 \\ \end{bmatrix}
\]

Następnie sformułowano prawo sterowania układu zamkniętego:
\[
\dot{x}=A_{closed}x+B_{closed}u
\]
Gdzie:
\[
A_{closed}=A_n-B_n K
\]
\[
B_{closed}=\begin{bmatrix}
  0\\
  0\\
  0
\end{bmatrix}
\]
Ponieważ badana jest stabilność układu zamkniętego, zmieniane będą
warunki początkowe bez podawania sygnału sterującego. Z tego powodu $B_{closed}$ może być kolumną zer.
Układ wytrącono z równowagi w postaci warunków początkowych wynoszących $\dot{x} = \begin{bmatrix} 1 & 0.5 & -0.5 \\ \end{bmatrix}$
\begin{figure}[H]
    \centering
    \includegraphics[width=0.7\linewidth]{img/zamkniety.png}
    \caption{Przebieg zmiennych stanu układu zamkniętego}
\end{figure}


\section{Ćwiczenie 5}

Na zajęciach numer 5 przedstawiono metody optymalizacji matematycznej, rozpatrywano dany problem:
\[
J=\int_{0}^{1}24x(t)t+2\dot{x}(t)^2-4tdt \rightarrow min
\]
z ograniczeniami 
\[
x(0)=1 \qquad x(1)=3
\]
Pomijając przekształcenia, ostatecznie otrzymano analitycznie optymalną trajektorię:
\[
x^*=t^3+t+1
\]

\newpage

\subsection{Kod wykorzystany do przeprowadzenia optymalizacji dynamicznej}

\begin{lstlisting}
m2=GEKKO()

m2.options.IMODE=6

n_points = 201
m2.time = np.linspace(0, 1, n_points)

x= m2.Var(value=1)

m2.fix_initial(x, val=1) # x(0)=1
m2.fix_final(x, val=3) # x(1)=3

J = m2.Var(value=0)
m2.fix_initial(J, val=0)

t = m2.Param(value=m2.time)

integrand = 24*x*t + 2*x.dt()**2 - 4*t #x.dt to x prim

m2.Equation(J.dt() == integrand)

J_f = m2.FV()
J_f.STATUS = 1 

m2.Connection(J_f, J, pos2='end')

m2.Obj(J_f)

m2.options.SOLVER=3

m2.solve(disp=False)

x_res=np.array(x.value)
J_res=J_f.value[0]

print(f'minimalna wartosc calki= {J_res:.6f}')
\end{lstlisting}

\begin{figure}[H]
  \centering
  \includegraphics[width=0.8\linewidth]{img/trajektorielab5.png}
\end{figure}

Po wyznaczeniu wektora wartości dla trajektorii wyznaczonej analitycznie i naniesieniu jej na jeden wykres wspólnie z trajektorią obliczoną przez GEKKO, można stwierdzić, że trajektorie te pokrywają się z nieznacznym błędem. Do wyznaczenia rzędu tego błędu może posłużyć wartość całkowego wskaźnika jakości:
\[
J_{GEKKO}= 32.729852
\] 
\[
J_{analitycznie}=\int_{0}^{1} 24x(t)t+2\dot{x}(t)^2-4t dt
\]
\begin{center}
jeśli:
\[
x(t)=t^3+t+1
\]
\[
\dot{x}(t)=3t^2+1
\]
to:
\end{center}
\[
J_{analitycznie}=\int_{0}^{1} 24(t^3+t+1)t + 2(3t^2+1)^2 -4t dt = \frac{162}{5} = 32,4
\]

Co daje błąd na poziomie:
\[
\frac{32,729852-32,4}{32,4}\cdot 100\%=1,01\%
\]

Biorąc pod uwagę błąd rzędu 1\%, można uznać metody optymalizacji dynamicznej przy użyciu 
GEKKO za bardzo wiarygodne.
\section{Ćwiczenie 6} 
\subsection{Rozważany układ}

\begin{figure}[H]
  \centering
  \includegraphics[width=0.5\linewidth]{img/ukladdopid.png}
  \caption{Rozważany układ elektryczny}
\end{figure}

Dla danego układu z parametrami $R_1=2\Omega$,$R_2=5\Omega$,$C_1=0.5F$,$L_1=2H$ i $L_2=0.5H$. Przyjmując zmienne w postaci $x=\begin{bmatrix}
  i_1 & i_2 & u_2 
\end{bmatrix}^T$, wyznaczonao równania stanu:
\[
\dot{x}=\begin{bmatrix}
  -\frac{R_1}{L_1} & 0 & -\frac{1}{L_1}\\
  0 & -\frac{R_2}{L_2} & \frac{1}{L_1}\\
  \frac{1}{C_1}& -\frac{1}{C_1}&0
\end{bmatrix}x +\begin{bmatrix}
  \frac{1}{L_1} \\ 0\\0
\end{bmatrix}u
\]

\subsection{Przebieg wyjścia układu z regulatorem PID}

Dla danego układu zaimplementowano regulator PID 
\begin{figure}[H]
  \centering
  \includegraphics[width=0.5\linewidth]{img/pid1.png}
  \caption{}
\end{figure}

Taki przebieg uzyskano dla empirycznie dobranych nastawów:
\[
\begin{bmatrix}
  k_P\\
  k_I\\
  k_D
\end{bmatrix}=\begin{bmatrix}
  10\\
  5\\
  5
\end{bmatrix}
\]

\subsection{Rola poszczególnych wzmocnień w regulatorze PID}

tutaj jakas pierdulka 

\subsection{Metoda Zieglera-Nicholsa}

Metoda Zieglera-Nicholsa polega na na odnalezieniu
krytycznej wartości wzmocnienia kp, dla której układ znajduje się na granicy stabilnośc. Wprowadzajac układ na granice stabilnosci zaczna wystepowac oscylacje. Dla poprawnego wyznaczenia wzmocnien według metody ZG należy wyznaczyc okres tych oscylacji.

\begin{figure}[H]
  \centering
  \includegraphics[width=0.5\linewidth]{img/granica.png}
  \caption{Przebieg wyjścia układu na granicy stabilności}
\end{figure}

Z wyznaczonymi parametrami ku i Tu można wynaczyc wzmocenienia dla poszczególnych czlonów różnych regulatorów

\begin{table}[H]
  \centering
\begin{tabular}{|l|l|l|l|}
\hline
Regulator & kp     & ki     & kd      \\ \hline
P         & 37,75  & -      & -       \\ \hline
PI        & 33,975 & 25,16  & -       \\ \hline
PD        & 60,4   & -      & 12,231  \\ \hline
PID       & 45,3   & 55,926 & 9,17325 \\ \hline
\end{tabular}
\end{table}

\begin{figure}[H]
  \centering
  \includegraphics[width=0.5\linewidth]{img/pidsymulacjezg.png}
\end{figure}

\subsection{Całkowe wskaźniki jakości}



\section{Ćwiczenie 7-8}
\subsection{Rozważany układ}

\begin{figure}[H]
  \centering
  \includegraphics[width=0.5\linewidth]{img/lqruklad.png}
  \caption{Rozważany układ RLC}
\end{figure}

Z przyjetymi prametrami $R=0.5\Omega$,$L=0.2H$ i $C=0.5F$, gdzie za zmienne stanu przyjeto $x=\begin{bmatrix}
  q_c & \dot{q_c}
\end{bmatrix}^T$. Wyprowadzono równania stanu

\[
\dot{x}=\begin{bmatrix}
  0 & 1\\
  -\frac{1}{LC} & -\frac{R}{L}
\end{bmatrix}x + \begin{bmatrix}
  0 \\
  \frac{1}{L}
\end{bmatrix}u
\]









\end{document}