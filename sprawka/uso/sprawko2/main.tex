\documentclass[11pt]{article}

\pdfobjcompresslevel=3    % compress PDF objects
\pdfcompresslevel=9       % compress streams more (0..9)
\pdfminorversion=4


\usepackage[utf8]{inputenc} % remove if using XeLaTeX or LuaLaTeX
\usepackage[T1]{fontenc}
\usepackage{polski}[babel]

% Layout & graphics
\usepackage[a4paper, total={7.5in, 9in}]{geometry}
\usepackage{graphicx}
\usepackage{wrapfig}              % images
\usepackage[dvipsnames,table]{xcolor} % single xcolor invocation (keeps table option)

% Math & symbols
\usepackage{amsmath}
\usepackage{amssymb}
\usepackage{gensymb}

% Typography & layout helpers
\usepackage{microtype}
\usepackage{float}                 % [H] placement
\usepackage{caption}
\usepackage{subcaption}            % modern subfigure support (do NOT load subfig)
\usepackage{multirow}
\usepackage{titlesec}

\usepackage{lmodern}
\usepackage{microtype}
\usepackage{listings}
\usepackage{xcolor}

\lstdefinestyle{MyStyle}{
    basicstyle=\ttfamily\small, % Font family and size
    keywordstyle=\color{blue}\bfseries, % Style for keywords
    commentstyle=\color{green!60!black}\itshape, % Style for comments
    %stringstyle=\color{red}, % Style for strings
    numberstyle=\tiny\color{gray}, % Style for line numbers
    backgroundcolor=\color{gray!10}, % Background color
    frame=single, % Add a frame around the code block
    breaklines=true, % Allows long lines to wrap
    language=Java, % Default language
    numbers=left, % Line numbers on the left
    showstringspaces=false % Don't show spaces in strings
}

\lstset{style=MyStyle}

\definecolor{xppblue}{RGB}{37, 150, 190}

\title{Sprawozdanie 2 z laboratorium \par Układy sterowania optymalnego}
\author{Jan Andrzejewski 159512}
\date{}

\begin{document}

\maketitle

\tableofcontents

\newpage

\section{Ćwiczenie 10-11}
\subsection*{Linearyzacja systemu nieliniowego wzgledem punktu pracy}


Rozważany jest obiekt nieliniowy.

\begin{figure}[H]
    \centering
    \includegraphics[width=0.6\linewidth]{img/wachadlorysunek.png}
    \caption{Schemat wachadła}
\end{figure}

Przyjmując $x_1=\theta$ i $x_2=\dot{\theta}$

\[
\dot{x}_1=x_2
\]
\[
\dot{x}_2=\frac{1}{J}u-\frac{d}{J}x_2-\frac{mgl}{J}sin(x_1)
\]

\subsubsection*{Linearyzacja wzgledem punktu $x_0=[0,0]$ $u=0$}

Przeprowadzono linearyzacje układu wzgledem punktu $x_0=[0,0]$ i $u=0$, ostatecznie otrzymując postać:
\[
\dot{\tilde{x}}=\begin{bmatrix}
    0 & 1\\
    -\frac{mgl}{J} & -\frac{d}{J}
\end{bmatrix}\tilde{x}+\begin{bmatrix}
    0\\
    \frac{1}{J}
\end{bmatrix}\tilde{u}
\]

gdzie $\dot{\tilde{x}}$, $\tilde{x}$, $\tilde{u}$ są równe $\dot{x}$, $x$ i $u$

\begin{figure}[H]
    \centering
    \includegraphics[width=0.6\linewidth]{img/predkoscwachadlo.png}
    \caption{placeholder}
\end{figure}

jeszcze jeden wykres dla wiekszego u 

Z przebiegów można wyciagnąc jednoznaczny wniosek. Linearyzacja wokoł punktu pracy dobrze opisuje model tylko w jego otoczeniu. Przyłożywszy sterowanie 

Zlinearyzowany model jest sterowanly ponieważ rząd macierzy Kalmana jest równy rzedowi macierzy A zlineryzowanego systemu 
\[
rank(A)=rank([B,AB])=2
\]

Oprocz tego że na sterowalnosc układu wskazuje rząd macierzy Kalmana, wskazuje na to postac sterowalna macierzy A.

\subsubsection*{Linearyzacja wzgledem punktu $x_0=[\frac{pi}{4},0]$ $u=45\sqrt{2}$}

Równania stanu

\[
\dot{\tilde{x}}=\begin{bmatrix}
    0 & 1\\
    -\frac{mgl}{J} & -\frac{d}{J}
\end{bmatrix}\tilde{x}+\begin{bmatrix}
    0\\
    \frac{1}{J}
\end{bmatrix}\tilde{u}
\]














\end{document}