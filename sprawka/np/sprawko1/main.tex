\documentclass[11pt]{article}
\usepackage{graphicx} % Required for inserting images
\usepackage[T1]{fontenc}
\usepackage[a4paper, total={7.5in, 9in}]{geometry}
\usepackage{amsmath}
\usepackage{amssymb}
\usepackage{gensymb}
\usepackage{float}
\usepackage{caption}
\usepackage{subcaption}
\usepackage{subfig}
\usepackage{multirow}
\usepackage{microtype}
\usepackage{titlesec}
\usepackage[dvipsnames]{xcolor}
\usepackage[table]{xcolor}

\definecolor{xppblue}{RGB}{37, 150, 190}

\begin
{document}

\begin
{table}[H]
\centering
\begin
{tabular}{|p{4cm}|p{8cm}|p{2cm}|}
\hline
\multicolumn{2}{|c|}{\cellcolor{xppblue}\textcolor[rgb]{1,1,1}{Politechnika Poznańska}}  & \multicolumn{1}{c|}{\multirow{3}{*}{\resizebox{14mm}{!}{\includegraphics{img/logo2.eps}}}}\\ 
\multicolumn{2}{|c|}{\cellcolor{xppblue}\textcolor[rgb]{1,1,1}{Wydział Automatyki, Robotyki i Elektrotechniki}} & \\ 
\multicolumn{2}{|c|}{\cellcolor{xppblue}\textcolor[rgb]{1,1,1}{Instytut Robotyki i Inteligencji Maszynowej}} & \\ 
\hline 
\multicolumn{1}{|c|}{Dz>AiR>Sem3} & \multicolumn{1}{c|}{Nazwa Przedmiotu} & \multicolumn{1}{c|}{2020/21 (s.zim.)} \\
\hline
Skład osobowy: \par B.Fabianski \par B.Fabianski & \textbf{temat ćwiczenia} & Data wyk.:\par 13.10.20\\
\hline
Grupa A11/1  & Ćwiczenie 1 & Zajęcia 1 \\
\hline
\end{tabular}
\end{table}	


\section{Introduction}

\end{document}