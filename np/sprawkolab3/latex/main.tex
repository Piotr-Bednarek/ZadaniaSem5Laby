\documentclass[a4paper,onecolumn,twoside,10pt]{article}%{mwrep}

\usepackage{times}
\usepackage[utf8x]{inputenc}
\usepackage[T1]{fontenc}
\usepackage[polish]{babel}
\usepackage{lmodern} %Type1-font for non-english texts and characters
\usepackage{setspace}
\usepackage{enumitem}
\usepackage{subcaption}

\usepackage{booktabs} % Dla ładniejszych linii poziomych (\toprule, \midrule)
\usepackage{colortbl} % Dla kolorowania wierszy
\usepackage{array}    % Użyteczne z p{...}


\usepackage[cmex10]{amsmath}

%% Packages for Graphics & Figures %%%%%%%%%%%%%%%%%%%%%%%%%%

%\usepackage{bmpsize}

\usepackage{graphicx} %%For loading graphic files
%\usepackage[pdf]{pstricks}
%\usepackage{pst-all}
%\usepackage{moredefs}
%\usepackage{auto-pst-pdf}
%\usepackage{auto-pst-pdf}
%\usepackage[crop=off]{auto-pst-pdf}

\usepackage{fancyhdr}
\usepackage{url}
\usepackage{float}
\usepackage{color}
\usepackage{xcolor}

\usepackage{multirow}

%\usepackage{epstopdf}
\definecolor{light-gray}{gray}{0.4}
\definecolor{mauve}{rgb}{0.88, 0.69, 1.0}
\definecolor{pakistangreen}{rgb}{0.0, 0.4, 0.0}
\definecolor{pearl}{rgb}{0.94, 0.92, 0.84}
\definecolor{whitesmoke}{rgb}{0.96, 0.96, 0.96}
\definecolor{gray-pp}{rgb}{0.13, 0.6, 0.82}

\usepackage{colortbl}
\usepackage{listings}
\lstset{
	basicstyle=\footnotesize\ttfamily,
	columns=fullflexible,
	frame=single,
	breaklines=true,
	numbers=left, stepnumber=2, numbersep=5pt,
	numberstyle=\tiny\color{gray},
	keywordstyle=\color{blue},
	commentstyle=\color{gray-pp},
	stringstyle=\color{mauve},
	backgroundcolor = \color{whitesmoke},
	breakatwhitespace=true,
	showspaces=false,                % show spaces everywhere adding particular underscores; it overrides 'showstringspaces'
	showstringspaces=false,          % underline spaces within strings only
	showtabs=false,                  % show tabs within strings adding particular underscores
	postbreak=\mbox{\textcolor{red}{$\hookrightarrow$}\space}
}



\cfoot{-~\thepage \textcolor{light-gray}{~| Strona~-}}

\hyphenpenalty=10000		% nie dziel wyrazów zbyt często
\clubpenalty=10000			% kara za sierotki
\widowpenalty=10000			% nie pozostawiaj wdów
\brokenpenalty=10000		% nie dziel wyrazów między stronami
\exhyphenpenalty=999999		% nie dziel słów z myślnikiem
\righthyphenmin=3			% dziel minimum 3 litery

\tolerance=4500
\pretolerance=250
\hfuzz=1.5pt
\hbadness=1450

\sloppy						% umacnia pozycję prawego marginesu

\setlength{\textwidth}{\paperwidth}
\addtolength{\textwidth}{-5cm}
\setlength{\textheight}{\paperheight}
\addtolength{\textheight}{-5cm}
\setlength{\oddsidemargin}{0cm}
\setlength{\evensidemargin}{0cm}
\topmargin -1.25cm
\footskip 1.4cm

\linespread{1.3}

\begin{document}
	\raggedbottom 
	%\input{hypernation}
	%\input{title-page}
	\setlength\extrarowheight{2pt}
	\begin{table}[ht]
		\centering
		%\resizebox{\textwidth}{!}{%
			\begin{tabular}{|p{5cm}|p{7cm}|p{2cm}|}
				\hline
				%\rowcolor{gray}
				%\multicolumn{3}{|c|}{\textcolor[rgb]{1,1,1}{Scalenie trzech kolumn}}\\
				
			\multicolumn{2}{|c|}{\cellcolor{gray-pp}\textcolor[rgb]{1,1,1}{Politechnika Poznańska}}  & \multicolumn{1}{c|}{\multirow{3}{*}{\resizebox{15mm}{!}{\includegraphics{PP_znak_konturowy_CMYK.pdf}}}}\\
				\multicolumn{2}{|c|}{\cellcolor{gray-pp}\textcolor[rgb]{1,1,1}{Wydział Automatyki, Robotyki i Elektrotechniki}} & \\ 
				\multicolumn{2}{|c|}{\cellcolor{gray-pp}\textcolor[rgb]{1,1,1}{Instytut Robotyki i Inteligencji Maszynowej}} & \\ 
				\hline 
				\multicolumn{1}{|c|}{Dz>AiR>Sem5} & \multicolumn{1}{c|}{Napędy przekształtnikowe (NP)} & \multicolumn{1}{c|}{2025/26 (s.zim.)} \\
				\hline
				\textbf{Skład osobowy:} \par Zuzanna Andrzejak 159522 \par Jan Andrzejewski 159512 \par Mateusz Banaszak 159416 \par Piotr Bednarek 159701 & 
				\textbf{Stany pracy silnika obcowzbudnego prądu stałego} 
				& Data wyk.:\par 08.01.2026\\
				\hline
				Grupa 1  & Ćwiczenie 3 & Zajęcia 3 \\
				\hline
			\end{tabular}%
			%}
	\end{table}	
	\setlength\extrarowheight{0pt}
	\vspace{1.5cm}
	\tableofcontents
	\newpage
	
	\section{Wprowadzenie}
	W poprzednim ćwiczeniu laboratoryjnym zdefiniowano model obwodowy obcowzbudnego silnika prądu stałego z komutatorem, przyjmując określone parametry w celu zrealizowania uproszczonego modelu. Celem kolejnego ćwiczenia było eksperymentalne wyznaczenie tych parametrów na podstawie pomiarów przeprowadzonych na rzeczywistym, identyfikowanym obiekcie \cite{ekurs}. Podczas analizy uwzględniono wektor parametrów modelu w postaci:
	\begin{equation}
		P^T = \left[ R_a,\; L_a,\; k_{\Phi},\; J_r,\; b_1 \right]
		\label{eq:parametry}
	\end{equation}
	
	
	Pomiary zrealizowano w sali C3 w budynku A22b, wyposażonej w stanowisko laboratoryjne z układem silnika umożliwiającym wymuszenie zadanej prędkości obrotowej wału. Stanowisko pomiarowe obejmowało ponadto multimetry, laboratoryjny zasilacz z ograniczeniem prądu oraz oscyloskop. W trakcie ćwiczenia wykorzystano również środowisko symulacyjne \texttt{MATLAB} do analizy zarejestrowanych danych pomiarowych oraz sporządzenia odpowiednich charakterystyk i wykresów.

	\vspace{1cm}

	\section{Charakterystyka sterowania}

	\begin{figure}[H]
		\centering
		\includegraphics[width=0.5\linewidth]{../wykresy/pdf/char_sterowania_1.pdf}
		\caption{}
	\end{figure}

	\begin{figure}[H]
		\centering
		\includegraphics[width=0.5\linewidth]{../wykresy/pdf/char_sterowania_2.pdf}
		\caption{}
	\end{figure}


	\vspace{1cm}

	\section{Charakterystyka mechaniczna}

	\begin{figure}[H]
		\centering
		\includegraphics[width=0.5\linewidth]{../wykresy/pdf/char_mechaniczna_150V.pdf}
		\caption{}
	\end{figure}

	\begin{figure}[H]
		\centering
		\includegraphics[width=0.5\linewidth]{../wykresy/pdf/char_mechaniczna_200V.pdf}
		\caption{}
	\end{figure}


	\vspace{1cm}

	\section{Hamowanie dynamiczne obcowzbudne}

	\vspace{1cm}

	\section{Hamowanie dynamiczne samowzbudne}

	\vspace{1cm}
	
	
	\vspace{1.5cm}
	
	\begin{thebibliography}{9}
		\bibitem{ekurs}
		Materiały do ćwiczenia laboratoryjnego dostępne na platformie eKursy, Politechnika Poznańska, \texttt{https://ekursy.put.poznan.pl/mod/folder/view.php?id=3022726}
		\bibitem{zawirski}
		K. Zawirski, J. Deskur, T. Kaczmarek, \textit{Automatyka napędu elektrycznego}, Wydawnictwo Politechniki Poznańskiej, 2012
		\bibitem{sidorowicz}
		J. Sidorowicz, \textit{Napęd elektryczny i jego sterowanie}, Oficyna Wydawicza Politechniki Warszawskiej, 1997
	\end{thebibliography}
	
	
	
	
\end{document}