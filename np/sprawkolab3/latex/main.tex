\documentclass[a4paper,onecolumn,twoside,10pt]{article}%{mwrep}

\usepackage{times}
\usepackage[utf8x]{inputenc}
\usepackage[T1]{fontenc}
\usepackage[polish]{babel}
\usepackage{lmodern} %Type1-font for non-english texts and characters
\usepackage{setspace}
\usepackage{enumitem}
\usepackage{subcaption}

\usepackage{booktabs} % Dla ładniejszych linii poziomych (\toprule, \midrule)
\usepackage{colortbl} % Dla kolorowania wierszy
\usepackage{array}    % Użyteczne z p{...}


\usepackage[cmex10]{amsmath}

%% Packages for Graphics & Figures %%%%%%%%%%%%%%%%%%%%%%%%%%

%\usepackage{bmpsize}

\usepackage{graphicx} %%For loading graphic files
%\usepackage[pdf]{pstricks}
%\usepackage{pst-all}
%\usepackage{moredefs}
%\usepackage{auto-pst-pdf}
%\usepackage{auto-pst-pdf}
%\usepackage[crop=off]{auto-pst-pdf}

\usepackage{fancyhdr}
\usepackage{url}
\usepackage{float}
\usepackage{color}
\usepackage{xcolor}

\usepackage{multirow}

%\usepackage{epstopdf}
\definecolor{light-gray}{gray}{0.4}
\definecolor{mauve}{rgb}{0.88, 0.69, 1.0}
\definecolor{pakistangreen}{rgb}{0.0, 0.4, 0.0}
\definecolor{pearl}{rgb}{0.94, 0.92, 0.84}
\definecolor{whitesmoke}{rgb}{0.96, 0.96, 0.96}
\definecolor{gray-pp}{rgb}{0.13, 0.6, 0.82}

\usepackage{colortbl}
\usepackage{listings}
\lstset{
	basicstyle=\footnotesize\ttfamily,
	columns=fullflexible,
	frame=single,
	breaklines=true,
	numbers=left, stepnumber=2, numbersep=5pt,
	numberstyle=\tiny\color{gray},
	keywordstyle=\color{blue},
	commentstyle=\color{gray-pp},
	stringstyle=\color{mauve},
	backgroundcolor = \color{whitesmoke},
	breakatwhitespace=true,
	showspaces=false,                % show spaces everywhere adding particular underscores; it overrides 'showstringspaces'
	showstringspaces=false,          % underline spaces within strings only
	showtabs=false,                  % show tabs within strings adding particular underscores
	postbreak=\mbox{\textcolor{red}{$\hookrightarrow$}\space}
}



\cfoot{-~\thepage \textcolor{light-gray}{~| Strona~-}}

\hyphenpenalty=10000		% nie dziel wyrazów zbyt często
\clubpenalty=10000			% kara za sierotki
\widowpenalty=10000			% nie pozostawiaj wdów
\brokenpenalty=10000		% nie dziel wyrazów między stronami
\exhyphenpenalty=999999		% nie dziel słów z myślnikiem
\righthyphenmin=3			% dziel minimum 3 litery

\tolerance=4500
\pretolerance=250
\hfuzz=1.5pt
\hbadness=1450

\sloppy						% umacnia pozycję prawego marginesu

\setlength{\textwidth}{\paperwidth}
\addtolength{\textwidth}{-5cm}
\setlength{\textheight}{\paperheight}
\addtolength{\textheight}{-5cm}
\setlength{\oddsidemargin}{0cm}
\setlength{\evensidemargin}{0cm}
\topmargin -1.25cm
\footskip 1.4cm

\linespread{1.3}

\begin{document}
	\raggedbottom 
	%\input{hypernation}
	%\input{title-page}
	\setlength\extrarowheight{2pt}
	\begin{table}[ht]
		\centering
		%\resizebox{\textwidth}{!}{%
			\begin{tabular}{|p{5cm}|p{7cm}|p{2cm}|}
				\hline
				%\rowcolor{gray}
				%\multicolumn{3}{|c|}{\textcolor[rgb]{1,1,1}{Scalenie trzech kolumn}}\\
				
			\multicolumn{2}{|c|}{\cellcolor{gray-pp}\textcolor[rgb]{1,1,1}{Politechnika Poznańska}}  & \multicolumn{1}{c|}{\multirow{3}{*}{\resizebox{15mm}{!}{\includegraphics{PP_znak_konturowy_CMYK.pdf}}}}\\
				\multicolumn{2}{|c|}{\cellcolor{gray-pp}\textcolor[rgb]{1,1,1}{Wydział Automatyki, Robotyki i Elektrotechniki}} & \\ 
				\multicolumn{2}{|c|}{\cellcolor{gray-pp}\textcolor[rgb]{1,1,1}{Instytut Robotyki i Inteligencji Maszynowej}} & \\ 
				\hline 
				\multicolumn{1}{|c|}{Dz>AiR>Sem5} & \multicolumn{1}{c|}{Napędy przekształtnikowe (NP)} & \multicolumn{1}{c|}{2025/26 (s.zim.)} \\
				\hline
				\textbf{Skład osobowy:} \par Zuzanna Andrzejak 159522 \par Jan Andrzejewski 159512 \par Mateusz Banaszak 159416 \par Piotr Bednarek 159701 & 
				\textbf{Stany pracy silnika obcowzbudnego prądu stałego} 
				& Data wyk.:\par 08.01.2026\\
				\hline
				Grupa 1  & Ćwiczenie 3 & Zajęcia 3 \\
				\hline
			\end{tabular}%
			%}
	\end{table}	
	\setlength\extrarowheight{0pt}
	\vspace{1.5cm}
	\tableofcontents
	\newpage
	
	\section{Wprowadzenie}
	Celem zadania jest zbadanie charakterystyk statycznych (sterowania, mechanicznej oraz hamowania dynamicznego) silnika prądu stałego, w tym zbadanie zgodności wpływu parametrów silnika (np. rezystancja twornika) na te charakterystyki.

	
	
	Pomiary zrealizowano w sali C3 w budynku A22b, wyposażonej w stanowisko laboratoryjne z układem silnika umożliwiającym wymuszenie zadanej prędkości obrotowej wału. Stanowisko pomiarowe obejmowało ponadto multimetry, laboratoryjny zasilacz z ograniczeniem prądu oraz oscyloskop. W trakcie ćwiczenia wykorzystano również środowisko symulacyjne \texttt{MATLAB} do analizy zarejestrowanych danych pomiarowych oraz sporządzenia odpowiednich charakterystyk i wykresów.

	\vspace{1cm}

	\section{Charakterystyka sterowania}

    \subsection{Metodyka pomiarów}
    Charakterystyka sterowania silnika to zależność prędkości obrotowej od napięcia sterującego w stanie statycznym. Pomiary przeprowadzono zgodnie z instrukcją prowadzącego na maksymalnym ograniczeniu prądowym, wyznaczonym zadajnikiem P2. W celu porównania wyznaczono charakterystykę z zerowym i z niezerowym momentem obciążenia. Moment obciążony realizowany jest za pomocą zewnętrznego napędu ALSPA. Zadajnik P1 ustawiono w pozycji 0, a P2 na stałą wartość obciążenia.\\
    W obu przypadkach kolejne punkty mierzono, zmieniając napięcie sterujące. 
    Wzór na prędkość obrotową silnika, co wiadomo z poprzednich jednostek laboratoryjnych można zapisać:\\
    \[
    \omega_r= \frac{U_a- R_aI_a}{k_{\phi}}
    \]
    gdzie $U_a$ to napięcie sterujące, $R_a$ rezystancja twornika, $I_a$ prąd twornika, a $k_{\phi}$ stałe wzmocnienie silnika.\\\\
    Wzór na moment obciążenia:\\
    \[
    T_e= k_{\phi}i_a
    \]
    \[
    i_a= \frac{T_e}{k_{\phi}}
    \]
    Żeby zobaczyć wpływ obciążenia na prędkość obrotową należy przekształcić wzór na prędkość, a następnie podstawić do niego moment obciążenia:
    \[
    \omega_r= \frac{U_a}{k_{\phi}}- \frac{R_ai_a}{k_{\phi}}
    \]
    \[
    \omega_r= \frac{U_a}{k_{\phi}}- \frac{R_aT_e}{{k_{\phi}}^{2}}
    \]
    \vspace{0.5cm}
    
    Z tej zależności można zaobserwować, że kształt charakterystyki sterującej bez i z momentem obciążenia nie zmieni się. Powinny być to przesunięte względem siebie równoległe proste. Charakterystyka z niezerowym obciążeniem będzie poniżej, ze względu na niższe ( o stałą wartość) prędkości dla tych samych napięć twornika.
    
    \subsection{Pomiary i opracowanie wyników}
    
	\begin{figure}[H]
		\centering
		\includegraphics[width=0.7\linewidth]{../wykresy/pdf/char_sterowania_zbiorcza.pdf}
		\caption{}
	\end{figure}

	Zgodnie z założeniami teoretycznymi charakterystyki sterowania są względnie równoległe. Rozbieżność policzymy porównując współczynniki kierunkowe prsotych, licząć błąd względny. Otrzymujemy rozbieżność na poziomie:\\
	\[
	\frac{5,16-5,05}{5,16}\cdot 100\% \approx 2,13\% 
	\]

	Jest to akceptowalne ze względu na błędy toru pomiarowego.


	\vspace{1cm}

	\section{Charakterystyka mechaniczna}
    
    \subsection{Metodyka pomiarów}
    Charakterystyka mechaniczna to zależność prędkości obrotowej od momentu hamującego $T_L$ przy stałej wartości napięcia uzwojeń twornika. Przeprowadzono pomiary dla dwóch wartości $U_a$ równych 150V i 200V. Dokonano tego, zmieniając obciążenie wprowadzone przez ALSPA pozostawiając napięcie twornika na stałym poziomie, a ograniczenie prądowe na maksymalnej, stałej wartości. Oczekujemy, że otrzymane przez nas proste po przeprowadzeniu regresji liniowej będą podobnie jak w poprzednim przypadku równoległe, ze względu na to, że różne napięcie w obu próbach wpłynie jedynie na przesunięcie prostej wyżej lub niżej.
    Wzór na prędkość obrotową pozostaje ten sam co przy charakterystyce sterowania z niezerowym momentem obciążenia, jednak zmieniła się zmienna, na której pracujemy. 

    
    \subsection{Pomiary i opracowanie wyników}
	\begin{figure}[H]
		\centering
		\includegraphics[width=0.7\linewidth]{../wykresy/pdf/char_mechaniczna_150V.pdf}
		\caption{}
	\end{figure}

	\begin{figure}[H]
		\centering
		\includegraphics[width=0.7\linewidth]{../wykresy/pdf/char_mechaniczna_200V.pdf}
		\caption{}
	\end{figure}


	\vspace{1cm}

    Tak jak widzimy z powyższych wykresów oraz wypisanych otrzymanych po regresji liniowej wzorów, proste są idealnie równoległe do siebie co zgadza się z założeniami teoretycznymi.
	\section{Hamowanie dynamiczne obcowzbudne}
    \subsection{Metodyka pomiarów}
    Hamowanie to polega na pracy maszyny jako prądnicy, w której obwód twornika jest odłączony od źródła napięcia ($U_a = 0$) i zamknięty przez rezystancję hamowania $R_{ad}$, natomiast obwód wzbudzenia zasilany jest w sposób ciągły. Dzięki utrzymaniu stałej wartości strumienia magnetycznego $k\phi$, wytworzony moment hamujący $T_H$ oraz prąd twornika są wprost proporcjonalne do prędkości kątowej wirnika. Zależność tę opisuje równanie:\\
    \[
    T_{H}= i_a\cdot k_{\phi}
    \]
   
    Pomiary rozpoczęto od odłączenia zasilania twornika z przekształtnika DML przy jednoczesnym zachowaniu zasilania obwodu wzbudzenia, co było kluczowe dla utrzymania stałej wartości strumienia magnetycznego $k\phi$2. Zaciski twornika połączono z zewnętrznym rezystorem dodatkowym $R_{ad}$ o wartości $40\Omega$.
    
    
    \subsection{Pomiary i opracowanie wyników}
    
    \begin{figure}[H]
		\centering
		\includegraphics[width=0.7\linewidth]{../wykresy/pdf/char_hamowania_obco.pdf}
		\caption{}
	\end{figure}

	\vspace{1cm}

	\section{Hamowanie dynamiczne samowzbudne}
    
    \subsection{Metodyka pomiarów}
    Hamowanie samowzbudne polega na pracy maszyny jako prądnicy obciążonej rezystancją $R_{ad}$, w której wzbudzenie następuje bez zewnętrznego źródła zasilania dzięki wykorzystaniu strumienia szczątkowego wynikającego z histerezy magnetycznej. Proces ten opiera się na dodatnim sprzężeniu zwrotnym, gdzie prąd płynący przez uzwojenie wzbudzenia zwiększa strumień, co z kolei indukuje wyższe napięcie w tworniku. 
    Przeprowadzenie pomiarów rozpoczęto od odłączenia zasilania twornika z przekształtnika DML i bezpośredniego przyłączenia uzwojenia wzbudzenia wraz z rezystorem obciążającym $40\Omega$ do zacisków maszyny. Wykorzystując układ ALSPA jako napęd pomocniczy pracujący w trybie regulacji wektorowej, stopniowo zwiększano prędkość obrotową zespołu, notując wartości prądu i mocy elektromagnetycznej, która jest w całości rozpraszana w formie ciepła na rezystancjach obwodu. Wyznaczono zależność momentu hamującego od prędkości obrotowej. W celu obliczenia go można użyć wzoru:\\
    \[
    T_{H}= \frac{P}{\omega_r}
    \]
    
    Tym razem nie wolno korzystać z zależności używanej przy hamowaniu obcowzbudnym, ponieważ w hamowaniu samowzbudnym prąd wzbudzenia jest generowany przez samą maszynę i zmienia się wraz z jej prędkością obrotową oraz napięciem indukowanym. Ponieważ strumień $k\phi$ jest bezpośrednio zależny od tego prądu, ulega on ciągłym zmianom zgodnie z nieliniową charakterystyką magnesowania obwodu magnetycznego.
	\subsection{Pomiary i opracowanie wyników}
    
    \begin{figure}[H]
		\centering
		\includegraphics[width=0.7\linewidth]{../wykresy/pdf/char_hamowania_samo.pdf}
		\caption{}
	\end{figure}

	\vspace{1cm}
	
	
	\vspace{1.5cm}
	
	\begin{thebibliography}{9}
		\bibitem{ekurs}
		Materiały do ćwiczenia laboratoryjnego dostępne na platformie eKursy, Politechnika Poznańska, \texttt{https://ekursy.put.poznan.pl/mod/folder/view.php?id=3022726}
		\bibitem{zawirski}
		K. Zawirski, J. Deskur, T. Kaczmarek, \textit{Automatyka napędu elektrycznego}, Wydawnictwo Politechniki Poznańskiej, 2012
		\bibitem{sidorowicz}
		J. Sidorowicz, \textit{Napęd elektryczny i jego sterowanie}, Oficyna Wydawicza Politechniki Warszawskiej, 1997
	\end{thebibliography}
	
	
	
	
\end{document}