\documentclass[11pt]{article}

\pdfobjcompresslevel=3    % compress PDF objects
\pdfcompresslevel=9       % compress streams more (0..9)
\pdfminorversion=4


\usepackage[utf8]{inputenc} % remove if using XeLaTeX or LuaLaTeX
\usepackage[T1]{fontenc}

% Layout & graphics
\usepackage[a4paper, total={7.5in, 9in}]{geometry}
\usepackage{graphicx}              % images
\usepackage[dvipsnames,table]{xcolor} % single xcolor invocation (keeps table option)

% Math & symbols
\usepackage{amsmath}
\usepackage{amssymb}
\usepackage{gensymb}

% Typography & layout helpers
\usepackage{microtype}
\usepackage{float}                 % [H] placement
\usepackage{caption}
\usepackage{subcaption}            % modern subfigure support (do NOT load subfig)
\usepackage{multirow}
\usepackage{titlesec}

\usepackage{lmodern}
\usepackage{microtype}

\usepackage{booktabs} % Dla ładniejszych linii poziomych (\toprule, \midrule)
\usepackage{colortbl} % Dla kolorowania wierszy
\usepackage{array}    % Użyteczne z p{...}

\definecolor{xppblue}{RGB}{37, 150, 190}

\begin{document}

\section{Przykładowa aplikacja silnika}

Silnik PZBb44b to maszynowy silnik prądu stałego o mocy znamionowej 1.5 kW,
zasilaniu 230 V oraz prądzie twornika 6.5 A. Pracuje przy prędkości
1450~rpm i posiada wzbudzenie własne o prądzie 0.4~A. Parametry te pozwalają
na wykorzystanie go w aplikacjach wymagających stabilnej prędkości obrotowej
i umiarkowanego momentu.

\subsection{Aplikacje}

Ze względu na moc i prędkość znamionową silnik może być stosowany w:
\begin{itemize}
    \item małych pompach wirowych,
    \item sprężarkach tłokowych o niskiej mocy,
    \item przenośnikach taśmowych,
    \item mieszadłach i mieszarkach przemysłowych,
    \item prostych napędach maszyn warsztatowych.
\end{itemize}

\subsubsection{Rozpoznanie rynku}

\begin{table}[h!]
\centering
\begin{tabular}{|p{2.7cm}|p{1.6cm}|p{1.8cm}|p{2.7cm}|p{3.2cm}|p{2cm}|}
\hline
\textbf{Model} & \textbf{Moc} & \textbf{Prędkość} & \textbf{Napięcie} & \textbf{Prądy} & \textbf{Moment}  \\ \hline

PZBb44b &
1.5 kW &
1450 rpm &
230 V (DC) &
6.5 A (tw.), 0.4 A (wzb.) &
$\approx 9.9$ Nm  \\ \hline

ABB M3AA &
1.5 kW &
1445--1450 rpm &
230/400/415 V, 3F, 50 Hz &
3.24--5.85 A &
$\approx 9.95$ Nm  \\ \hline

Siemens SIMOTICS GP &
1.5 kW &
$\approx 1450$ rpm &
230/400 V lub 400/690 V, 3F &
$\approx 8.4$ A (zależnie od wersji) &
---  \\ \hline

WEG W22 &
1.5 kW &
1450--1500 rpm &
220--240/380--415 V, 3F &
--- &
---  \\ \hline

Maedler / SM-I IEC 90L &
1.5 kW &
1430--1450 rpm &
230/400 V, 3F &
--- &
---  \\ \hline

\end{tabular}
\end{table}

\begin{table}[h!]
    \centering
    \caption{Porównanie parametrów silników elektrycznych (Uproszczony kod)}
    \vspace{0.2cm}
    
    \begin{tabular}{p{3cm} p{1.2cm} p{2cm} p{3.5cm} p{3cm} p{2cm}}
        \toprule % Najgrubsza linia na górze
        \textbf{Model} & \textbf{Moc} & \textbf{Prędkość} & \textbf{Napięcie} & \textbf{Prądy} & \textbf{Moment} \\ 
        \midrule % Średnia linia oddzielająca nagłówek
        
        PZBb44b & 1.5 kW & 1450 rpm & 230 V (DC) & 6.5 A (tw.), 0.4 A (wzb.) & $\approx 9.9$ Nm \\ 
        \rowcolor[gray]{0.95} % Szary kolor wiersza
        ABB M3AA & 1.5 kW & 1445--1450 rpm & 230/400/415 V, 3F, 50 Hz & 3.24--5.85 A & $\approx 9.95$ Nm \\ 
        Siemens SIMOTICS GP & 1.5 kW & $\approx 1450$ rpm & 230/400 V lub 400/690 V, 3F & $\approx 8.4$ A (zależnie od wersji) & --- \\ 
        \rowcolor[gray]{0.95} % Szary kolor wiersza
        WEG W22 & 1.5 kW & 1450--1500 rpm & 220--240 / 380--415 V, 3F & --- & --- \\ 
        Maedler / SM-I IEC 90L & 1.5 kW & 1430--1450 rpm & 230/400 V, 3F & --- & --- \\ 
        
        \bottomrule % Najgrubsza linia na dole
    \end{tabular}
\end{table}

\begin{table}[h!]
    \centering
    \caption{Porównanie parametrów znamionowych i symulacyjnych}
    \begin{tabular}{l c c}
        \toprule
        \textbf{Parametr} & \textbf{Znamionowe} & \textbf{Symulacja} \\ 
        \midrule
        Napięcie (V) & 230 & 230 \\ 
        Prędkość (rpm) & 1450 & 1131 \\ 
        Prąd (A) & 6,5 & 5,24 \\ 
        Moc (W) & 1500 & 1200 \\ 
        \bottomrule
    \end{tabular}
\end{table}

Silniki te stanowią porównywalne konstrukcje pod względem mocy i prędkości,
przez co mogą być wykorzystane jako modele referencyjne w procesie symulacji
i doboru parametrów zastępczych. Główne różnice pomiędzy silnikiem PZBb44b a pozostałymi silnikami
branymi pod uwagę wynikają przede wszystkim z odmiennych metod zasilania, a co za tym idzie
z charakterystycznych sposobów regulacji prędkości oraz kształtowania
momentu. Silnik PZBb44b jako maszyna prądu stałego z wzbudzeniem własnym
utrzymuje prędkość zależną od napięcia twornika oraz strumienia
wzbudzenia, co umożliwia płynną i szeroką regulację obrotów bez konieczności
stosowania dodatkowych układów energoelektronicznych. Z kolei silniki
indukcyjne AC pracują przy prędkości wyznaczonej przez częstotliwość sieci
i liczbę biegunów, a zmiana prędkości wymaga użycia falownika. W efekcie
każda z tych maszyn cechuje się odmiennym zachowaniem dynamicznym,
sprawnością regulacji oraz zakresem możliwej kontroli momentu, co przekłada
się na praktyczne zastosowania i sposób integracji z układem napędowym.

\end{document}